\newcommand{\roboSavannahTokens}{$N$}
\section{RoboSavannah}
\label{sec:test_robo_savannah}

During the RoboSavannah, the robots go into the wild and mix with the audience. The robots must follow and guide an untrained member of the audience through the RoboCup venue. 

\subsection{Setup}
The robots are lined up in the direct vicinity of the arena, in a publicly accessible area.

\subsection{Task}
The robots must follow and guide a person around the RoboCup venue. Robots are run in parallel as many as possible at a time, limited by safety guidance. 

\begin{itemize}
 \item  At any moment when the robot is not following or guiding, it can be commanded to start doing so. The robot may tell it is available for such a task to the audience.
	In this state it may announce what to say to start/stop following and guiding. 
 \item 	After receiving the command, the robot must follow/guide the person until it is commanded to stop. 
 \item \textbf{Following: }	The person may take the robot to any publicly and robot acccessible location within the venue. 
	If the robot has stopped following (for any reason) the person, the robot must indicate that its available for a command again and wait for a new command. 
 \item 	\textbf{Guiding: } 	The robot guides the person to the original starting location, i.e the vicinity of the arena. 
 \item 	This repeats until the time is up.
\end{itemize}

While following or guiding, it may be that the robot and/or the human is blocked or someone crosses between the follower and guider. 

\subsection{Jury \& evaluation}
General audience evaluates the following \& guiding performance and appearance of the robots. 
Each member of the audience who enters the corridor will receive \roboSavannahTokens tokens which will be given to their top favorite robots on both: appearance and performance. 
Earned tokens are normalized regarding the maximum amount of tokens given to a robot per category. 
Therefore scoring in each category is proportional to the amount of tokens per category.

\subsection{Security Concerns}
Robots mixing with the general audience has been avoided previosuly at RoboCup@Home, but to make robots useful, some mixing will be necessary.

\begin{itemize}
 \item Robots not deemed safe for this challenge by a team member or RoboCup@Home comittee member is excluded from this challenge.
 \item A team member will accompany the robot at all times in close distance in order to push the emergency if needed. 
 \item A referee will also accompany the robot when it is following or guiding persons. 
 \item Robot may not collide with any person or other robot. 
 \item The audience may not touch the robot. 
\end{itemize}

\paragraph*{Important Note:} If the robot is headed on a collision course toward a person, robot or anything else, it must be stopped immediately. 

\subsection{Additional rules and remarks}
\begin{itemize}
\item \textbf{Protagonist robots:} Robots must be able to perform autonomously during the test. 
  Team members are only allowed to interact with the audience in order to give very brief instructions, but this must be kept to a minimum.
  In the case that team members are interacting with the audience too much, the team can be disqualified. 
  This will be judged by the accompanying referee.

\item \textbf{Restart:} There is no restart limit nor penalty for restarting a robot. 
  However, it is important to note that this test is essentially scored by the general public, and it is reasonable to expect that the audience will not be attracted to a robot being constantly fixed. 

\item \textbf{Focus on performance:} We encourage teams to show the best of @home in the RoboSavannah, which can be achieved by having well-functioning and clear but polite robots.

\item \textbf{Charging:} It is allowed to the team to change robot's batteries or to use a charging station during the RoboSavannah, however, all equipment must be at the arena.

\item \textbf{Gifts:} Robots and team member are \emph{not} allowed to hand out gifts as part of the RoboSavannah challenge. 

\item \textbf{Language:} RoboCup is an international competition. This means all robots have to interact with the audience in English.

\item \textbf{Asking for passage:} The robot is allowed to (gently) ask individual persons to step aside, but it is not allowed to blindly shout at groups of people.
\end{itemize}

\subsection{Data recording}
No data required to be recorded during the RoboSavannah challenge.

\subsection{OC instructions}

2h before test:
\begin{itemize}
\item Announce to teams the order in which the robots must start.
\end{itemize}

During the test
\begin{itemize}
\item Provide tokens to the audience.
\item Stop unattended robots (by pressing the emergency button).
\end{itemize}

\subsection{Score Sheet}
The maximum time for this test is 60 minutes.

Robots are scored on performance while following and guiding and on design. 
The audience can awards tokens for what they elect to be the \textbf{Most functional robot} and the \textbf{Best looking robot}. 
\textit{TC} and \textit{EC} evaluates technical performance.

\renewcommand{\dataRecordingBonus}{false}
\begin{scorelist}
	\scoreheading{Audience rating}
	\scoreitem{25}{Robot's appearance}
	\scoreitem{25}{Robot's performance}

	\scoreheading{Technical performance (max 50)}
	\scoreitem{20}{Following}
	\scoreitem{20}{Guiding}
	\scoreitem{10}{Interaction with audience}

	\setTotalScore{100}
\end{scorelist}

\paragraph*{Normalization}: Teams get score proportional to the best team of the category:

$$\text{score for this team} = 25 \times \frac{t_{this}}{t_{best}}$$
where $t_{this}$, $t_{best}$ is the number of tokens received by this team, and the number of tokens received by the best team.

\renewcommand{\dataRecordingBonus}{true}
% Local Variables:
% TeX-master: "Rulebook"
% End:



% Local Variables:
% TeX-master: "Rulebook"
% End:
