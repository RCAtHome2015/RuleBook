% %% %%%%%%%%%%%%%%%%%%%%%%%%%%%%%%%%%%%%%%%%%%%%%%%%%%%%%%%%%
%
% External Devices
%
% %% %%%%%%%%%%%%%%%%%%%%%%%%%%%%%%%%%%%%%%%%%%%%%%%%%%%%%%%%%
\section{External devices}
\label{rule:robot_external_devices}
Everything which is not part of the robot is considered an \iterm{external device}.
All external devices must be authorized by the \iaterm{Technical Committee}{TC} during the \iterm{Robot Inspection} test (see~\refsec{sec:robot_inspection}).
The \iaterm{Technical Committee}{TC} specifies whether an external device can be used freely, under referee supervision, and its impact on scoring.
In general, external devices must be removed quickly after the test.

\noindent \textbf{Remark:} The use of \iterm{wireless devices} is strictly prohibited. \iterm{External microphones}, hand microphones, and headsets are not allowed in OPL and it use is discouraged in DSPL and SSPL.

\subsection{On-site external computing}
Computing resources that are not physical attached to the robot are considered \iterm{external computing resources}.
The use of up to 5 external computing resources is allowed, but only through the arena network (see \refsec{rule:scenario_wifi}) and with the previous approval of the \iaterm{Technical Committee}{TC}.
Teams must announce the use of any external computing resource at least 1 month before the competition to the \iaterm{Technical Committee}{TC}.

External Computing Devices must be placed in the \iaterm{\textbf{E}xternal \textbf{C}omputing \textbf{R}esource \textbf{A}rea}{ECRA} which is announced by the \iaterm{Technical Committee}{TC} during setup days.
A switch connected to the arena wireless network will be available to teams in the ECRA.
It is strictly forbidden to connect any kind of device or peripheral (e.g.~screens, mouses, keyboards, etc.) to the computers in the ECRA during the competition.

A maximum of two laptops and two people from different teams is allowed at any time in the ECRA.
Teams using laptops as External Computing Devices must remove the device immediately after the test.
Once a test has started, all people must stay at least 1 meter from the ECRA.
Interacting with computers in the ECRA after the Referee has given the start signal will cause the immediate disqualification of the team.

\noindent \textbf{Remark:} Robot operation must be able to operate safely when \iterm{external computing resources} are unavailable.



% On-line devices
\subsection{On-line external computing}
\label{rule:robot_external_computing_online}
Robots are allowed to use \enquote{Cloud services}, \enquote{Internet API's}, and any other type of \iterm{external computing resource}.
Same restrictions for on-site external computing resources apply.

\noindent \textbf{Remark:} The competition organization doesn't guarantee or take any responsibility regarding the availability or reliability of neither the network nor Internet connection.
Teams' use of external computing resources is at their own risk.



% DSPL laptop
\subsection{Official Standard Laptop for DSPL}
\label{rule:osl_dspl}

In the Domestic Standard Platform League, teams must use the \iaterm{Official Standard Laptop}{OSL} connected to the Toyota HSR via Ethernet cable, safely located in the TOYOTA HSR \iterm{Mounting Bracket} provided by TOYOTA for this purpose.

Any laptop fitting inside the TOYOTA HSR \iterm{Mounting Bracket} is allowed, regardless of its technical specification.
All competing robots must have mounted an OSL, whether they use it or not, so all TOYOTA HSRs have the same load restrictions.

\subsubsection{External Wi-Fi adapter}
As described in section 6.2.2.5.6 of the Toyota HSR manual, an external Wi-Fi adapter can be used to increase stability and/or bandwidth of the network connection. As teams are allowed to use the network adapter of the \iterm{OSL}, teams may also chose to use an external Wi-Fi adapter.

No constraints are applied regarding the Wi-Fi adapter specifications. However to be used on the Toyota HSR, the Wi-Fi adapter should be USB-powered. To allow wired network connections between the internal computer, the \iterm{OSL} and the external Wi-Fi adapter, a (USB-powered) switch is allowed.

\noindent \textbf{Remark:} The usage of the network adapter of the \iterm{OSL} and/or the external Wi-Fi adapter doesn't allow for more than \emph{one} Wi-Fi connection per robot to the \iterm{arena network}.

% Local Variables:
% TeX-master: "../Rulebook"
% End:
