%%%%%%%%%%%%%%%%%%%%%%%%%%%%%%%%%%%%%%%%%%%%%%%%%%%%%%%%%
\section{Organization of the competition}
\label{sec:procedure_during_competition}

\subsection{Stage system}\label{rule:stages}

The competition features a \iterm{stage system}. It is organized in two stages each consisting of a number of specific tasks. It ends with the \iterm{Finals}.


\begin{enumerate}
	\item \textbf{Robot Inspection:} For security, robots are inspected during setup days.
	A robot must pass \iterm{Robot Inspection} test (see~\refsec{sec:robot_inspection}) in order to compete.

	\item \textbf{Stage~I:} The first days of the competition called \iterm{Stage~I}.
	All qualified teams can participate in \iterm{Stage~I}.
	The same task can be performed multiple times (See~\refsec{rule:score_system}).

	\item \textbf{Stage~II:} The best \emph{50\% of teams}\footnotemark (after Stage~I) advance to \iterm{Stage~II}.
	Here, tasks require more complex abilities or combinations of abilities.\\
	\footnotetext{If the total number of teams is less than 12, up to 6 teams may advance to Stage~II}

	\item \textbf{Final demonstration:} The best \emph{two teams} of each league, the ones with the highest score after Stage~II, advance to the final round.
	The final round features only a single task integrating all tested abilities.
	In order to participate in the Finals, a team must have solved at least one task of the Stage~II.
\end{enumerate}

In case of having no considerable score deviation between a team advancing to the next stage and a team dropping out, the TC may announce additional teams advancing to the next stage.

Each \iterm{stage} comprehends a set of tasks grouped in two thematic scenarios.
The \iterm{Housekeeper} scenario features tasks related to cleaning, organizing, and maintenance.
The \iterm{Party Host} scenario focuses on providing general assistance during a party by attending the needs of the guests.


%%%%%%%%%%%%%%%%%%%%%%%%%%%%%%%%%%%%%%%%%%%%%%%%%%%%%%%%%
\subsection{Schedule}
\label{rule:schedule}

\begin{enumerate}
	\item \textbf{Thematic scenario blocks:} Two blocks are scheduled per day, lasting between two and three hours.
	All teams have assigned at least 2 \iterm{testing slots} per block in which they can test any \iterm{task} of their choice from the block's assigned scenario.

	\item \textbf{Slots:} In principle, all teams get the same amount of \iterm{testing slots} with a minimum of 2 per block.
	If there is sufficient unoccupied remaining time for all the teams in a league, a referee can open an extra testing slot, allowing an extra run for all.

	\item \textbf{Tests:} Teams must inform the OC in advance which tasks will be tested in each block.
	Only one task can be attempted per test slot.

	\item \textbf{Participation is default:} Teams have to indicate to the \iaterm{Organizing Committee}{OC} when they are \emph{skipping} a testing slot. Without such indication, they may receive a penalty when not attending (see~\refsec{rule:not_attending}).
\end{enumerate}

% Please add the following required packages to your document preamble:
% \usepackage[table,xcdraw]{xcolor}
% If you use beamer only pass "xcolor=table" option, i.e. \documentclass[xcolor=table]{beamer}
\begin{table}[h]
	\centering\small
	\newcommand{\teams}[3]{%
		\tiny
		\begin{tabular}{c}%
			\textit{Test slot 1, team $#1$}\\
			\textit{Test slot 2, team $#2$}\\
			$\vdots$\\
			\textit{Test slot $n$, team $#3$}\\
		\end{tabular}
	}
	\newcommand{\wcell}[2]{%
		\parbox[c]{2.5cm}{%
			\vspace{#1}%
			\centering%
			#2%
			\vspace{#1}%
		}%
	}
	\newcommand{\cell}[1]{\wcell{0.2\baselineskip}{#1}}
	% \newcommand{\mr}[1]{\multirow{2}{*}{#1}}


	\begin{tabular}{
		>{\centering\arraybackslash}m{2.5cm}|%
		>{\columncolor[HTML]{9AFF99}}c |%
		>{\columncolor[HTML]{9AFF99}}c |%
		>{\columncolor[HTML]{CBCEFB}}c |%
		>{\columncolor[HTML]{FF8D27}}c  %
	}
	\multicolumn{1}{ c }{}
		& \multicolumn{1}{ c }{\cellcolor{white} Day 1 }
		& \multicolumn{1}{ c }{\cellcolor{white} Day 2 }
		& \multicolumn{1}{ c }{\cellcolor{white} Day 3 }
		& \multicolumn{1}{ c }{\cellcolor{white} Day 4 }
		\\\hhline{~---~}

	\cell{Block 1\\\footnotesize(9:00--12:00)}
		& \cell{Housekeeper\\\teams{i}{j}{i}}
		& \cell{Housekeeper\\~\\Party Host}
		& \cell{Restaurant}
		& \cellcolor{white}
		\\\hhline{~----}



	\multicolumn{1}{ c }{}
		& \multicolumn{3}{ c }{\wcell{0.5\baselineskip}{\color{gray}Lunch}}
		& \multicolumn{1}{|c|}{\cellcolor[HTML]{FF8D27}\cell{\textbf{Finals}}}
		\\\hhline{~----}

	\cell{Block 2\\\footnotesize(14:00--17:00)}
		& \cell{Party Host\\\teams{i}{k}{i}}
		& \cellcolor[HTML]{CBCEFB}\cell{Party Host}
		& \cell{Housekeeper}
		& \cellcolor{white}
		\\\hhline{~---~}

	\multicolumn{1}{ c }{}
		& \multicolumn{1}{ c }{\wcell{0.5\baselineskip}{\color[HTML]{029734}Stage 1}}
		& \multicolumn{1}{ c }{\cellcolor{white}}
		& \multicolumn{1}{ c }{\wcell{0.5\baselineskip}{\color[HTML]{6668e5}Stage 2}}\\
	\end{tabular}

	\caption{Example schedule.
		Each team has assigned at least two test slots in every block.
		At least two blocks are scheduled per day with an assigned theme.
		A team can choose a different task in each test, meaning at least 4 different tests per stage.
	}
	\label{tbl:schedule}
\end{table}

\noindent \textbf{Remark:} The \iaterm{Organizing Committee}{OC} announces the schedule during the setup days (see Table \ref{tbl:schedule}).

\subsection{Score system}
\label{rule:score_system}
Each task has a main objective and a set of scoring bonuses.
To score in a test, a team must successfully accomplish the main objective of the task; bonuses are not considered otherwise.
Overall scoring is calculated as the sum of the maximum score obtained in each ability.

The \iaterm{score system} has the following constrains
\begin{enumerate}

	\item \textbf{Stage~I:} The maximum total score per task in \iterm{Stage~I} is \scoring{1000 points}.

	\item \textbf{Stage~II:} The maximum total score per task in \iterm{Stage~I} is \scoring{2000 points}.

	\item \textbf{\iterm{Finals}:} Final score is normalized and a special evaluation is used.

	\item \textbf{Minimum score:} The minimum total score per test in \iterm{Stage~I} and \iterm{Stage~II} is \scoring{0 points}.
	Teams cannot receive negative points.

	\item \textbf{Penalties:} An exception to \emph{minimum score} rule are penalties.
	Both penalties for not attending (see~\refsec{rule:not_attending}) and extraordinary penalties (see~\refsec{rule:extraordinary_penalties}) can cause a total negative score.
\end{enumerate}




% Local Variables:
% TeX-master: "../Rulebook"
% End:
