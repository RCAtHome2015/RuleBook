%%%%%%%%%%%%%%%%%%%%%%%%%%%%%%%%%%%%%%%%%%%%%%%%%%%%%%%%%%%%%%%%%%%%%%%%%%%%%%%
%%
%%          $Id: Rulebook.tex 2014-12-12 balkce $
%%    author(s): RoboCupAtHome Technical Committee(s)
%%  description: introduction to RoboCupAtHome
%%
%%%%%%%%%%%%%%%%%%%%%%%%%%%%%%%%%%%%%%%%%%%%%%%%%%%%%%%%%%%%%%%%%%%%%%%%%%%%%%%
\documentclass[11pt, twoside, openright, a4paper, chapterprefix]{scrbook}
\usepackage[inner=2.5cm, outer=2.5cm, top=4cm, bottom=4cm]{geometry}

%%% PACKAGES %%%%%%%%%%%%%%%%%%%%%%%%%%%%%%%%%%%%%%%%%%%%%%%%%%%%%%%%%%%%%%%%%%
%%%%%%%%%%%%%%%%%%%%%%%%%%%%%%%%%%%%%%%%%%%%%%%%%%%%%%%%%%%%%%%%%%%%%%%%%%%%%%%
%%
%%          $Id: packages.tex 385 2013-02-12 21:53:10Z holz $
%%    author(s): RoboCupAtHome Technical Committee(s)
%%  description: List of packages for the RoboCupAtHome rulebook
%%
%%%%%%%%%%%%%%%%%%%%%%%%%%%%%%%%%%%%%%%%%%%%%%%%%%%%%%%%%%%%%%%%%%%%%%%%%%%%%%%
% \usepackage{soul}

\usepackage[utf8x]{inputenc}
\usepackage[english]{babel}
\usepackage{amsmath,amssymb,amsfonts}
% \usepackage[nice]{nicefrac}
\usepackage{siunitx}
\usepackage{graphicx}
\usepackage{multicol}
\usepackage{verbatim}
\usepackage{fancyhdr}

% \usepackage{color}
\usepackage{xcolor}
\usepackage{colortbl}
% \usepackage{epsfig}
\usepackage{makeidx} % This one causes scoresheets not to setle
% \usepackage{lscape}
% \usepackage{picinpar}

\usepackage{./styles/tweaklist}

\usepackage{enumerate}
\usepackage{paralist}
\usepackage{multirow}
\usepackage{hhline}
\usepackage{pgffor}
% \usepackage{array}

\usepackage{nameref}
\usepackage{varioref}
\usepackage{hyperref}
\usepackage[noabbrev,nameinlink]{cleveref}
\usepackage{tabularx}
\usepackage{xspace}
\usepackage{csquotes}
\usepackage[inline]{enumitem}

%\usepackage{times}
%\usepackage{helvet}
%\usepackage{courier}

% \usepackage{url}
\usepackage{caption}
% \usepackage{epstopdf}
\usepackage{subfig}
\usepackage{float}
\usepackage{wrapfig}
% \usepackage{xfrac}

% \usepackage[titletoc]{appendix}
% \usepackage{enumitem}
% \usepackage{mathtools}
% \usepackage{gensymb}

% Required by scoresheets
\usepackage{calc}
\usepackage{ifthen}
\usepackage{environ}
\usepackage{wasysym}
\usepackage{chngpage}

% Local Variables:
% TeX-master: "../Rulebook"
% End:

\usepackage[titletoc]{appendix}
\usepackage{enumitem}
\usepackage{mathtools}
\usepackage{gensymb}
\setlist{noitemsep}

%%% SubfigureSetup %%%%%%%%%%%%%%%%%%%%%%%%%%%%%%%%%%%%%%%%%%%%%%%%%%%%%%%%%%%%
%\renewcommand{\subfigtopskip}{5pt}        % default is 10pt
%\renewcommand{\subfigbottomskip}{5pt}     % default is 10pt
%\renewcommand{\subfigcapskip}{3pt}        % default is 10pt
%\renewcommand{\subfigcapmargin}{7pt}      % default is 10pt

%%% TweakList-Setup %%%%%%%%%%%%%%%%%%%%%%%%%%%%%%%%%%%%%%%%%%%%%%%%%%%%%%%%%%%
\renewcommand{\itemhook}{%                 % modify itemize-spacing
  \setlength{\topsep}{2pt}%
  \setlength{\partopsep}{1pt}%
  \setlength{\itemsep}{-1pt}%
}
\renewcommand{\enumhook}{%                 % modify enumerate-spacing
  \setlength{\topsep}{2pt}%
  \setlength{\partopsep}{1pt}%
  \setlength{\itemsep}{-1pt}%
}
\renewcommand{\descripthook}{%             % modify description-spacing
  \setlength{\topsep}{2pt}%
  \setlength{\partopsep}{1pt}%
  \setlength{\itemsep}{-1pt}%
}

\setkomafont{title}{\normalfont}
\setkomafont{sectioning}{\normalfont\bfseries}
\addtokomafont{caption}{\small}
\setkomafont{captionlabel}{\small\bfseries}
\setkomafont{descriptionlabel}{\normalfont\bfseries}
\renewcommand*{\chapterformat}{\LARGE{Chapter \thechapter}}

%%% MACROS %%%%%%%%%%%%%%%%%%%%%%%%%%%%%%%%%%%%%%%%%%%%%%%%%%%%%%%%%%%%%%%%%%%%
\input{./setup/active_version.tex}
\graphicspath{{\YEAR/}{./images/}}
%%%%%%%%%%%%%%%%%%%%%%%%%%%%%%%%%%%%%%%%%%%%%%%%%%%%%%%%%%%%%%%%%%%%%%%%%%%%%%%
%%
%%          $Id: macros.tex 399 2013-02-14 20:24:02Z holz $
%%    author(s): RoboCupAtHome Technical Committee(s)
%%  description: Macros for the RoboCupAtHome rulebook
%%
%%%%%%%%%%%%%%%%%%%%%%%%%%%%%%%%%%%%%%%%%%%%%%%%%%%%%%%%%%%%%%%%%%%%%%%%%%%%%%%

%%%%%%%%%%%%%%%%%%%%%%%%%%%%%%%%%%%%%%%%%%%%%%%%%%%%%%%%%%%%%%%%%%%%
% Macros for generating score sheets for RoboCup@Home              %
% to be used in the rulebook or during the competition             %
%                                                                  %
% Author: Dirk Holz & David Gossow                                 %
% Modif : Mauricio Matamoros                                       %
% $Id: macros_score_sheets.tex 429 2013-04-30 10:09:55Z holz $     %
%%%%%%%%%%%%%%%%%%%%%%%%%%%%%%%%%%%%%%%%%%%%%%%%%%%%%%%%%%%%%%%%%%%%
% chktex-file 1
% chktex-file 15
% chktex-file 21
% chktex-file 35
% chktex-file 44

% %%% %%%%%%%%%%%%%%%%%%%%%%%%%%%%%%%%%%%%%%%%%%%%%%%%%%%%%%%%%%%%%%
%                                                                  %
% GLOBAL OPTIONS                                                   %
%                                                                  %
% %%% %%%%%%%%%%%%%%%%%%%%%%%%%%%%%%%%%%%%%%%%%%%%%%%%%%%%%%%%%%%%%%

% Set \shortScoresheet to true for the rulebook version
% Set \shortScoresheet to false for the referee's scoresheet
\newcommand{\shortScoresheet}{true}

% The global number of attempts per test
\newcommand{\attempts}{3}

% Sets the total penalty for not showing up
\newcommand{\notattendingpenalty}{500}

% Set to true to display the "Using start button" penalty item
\newcommand{\startbuttonpenalized}{true}

% Sets the total penalty for not using start button instead of door
\newcommand{\startbuttonpenalty}{100}

% Set to true to display the the outstanding performance bonus item
\newcommand{\outstandingPerformanceBonus}{true}

% Percentage of the outstanding performance bonus (ommit % symbol)
\newcommand{\outstandingPerformanceBonusPercentage}{10}

% Set to true to display the data recording bonus item
\newcommand{\dataRecordingBonus}{false}

% Percentage of the data recording bonus (ommit % symbol)
\newcommand{\dataRecordingBonusPercentage}{10}

% Sets the name of the column for referee scoring when \attempts=1
\newcommand{\singleTryColumnCaption}{Single try}

% Sets the first column's name for referee scoring when \attempts=2
\newcommand{\firstTryColumnCaption}{First try}

% Sets the second column's name for referee scoring when \attempts=2
\newcommand{\secondTryColumnCaption}{Restart}

% Sets the name of the column for referee scoring when \attempts=1
\newcommand{\firstColumnCaption}{Action}

% Sets the second column's name for referee scoring when \attempts=2
\newcommand{\secondColumnCaption}{Score}

% %%% %%%%%%%%%%%%%%%%%%%%%%%%%%%%%%%%%%%%%%%%%%%%%%%%%%%%%%%%%%%%%%
%                                                                  %
% USAGE                                                            %
%                                                                  %
% %%% %%%%%%%%%%%%%%%%%%%%%%%%%%%%%%%%%%%%%%%%%%%%%%%%%%%%%%%%%%%%%%
%
% A scoresheet must be in a separated tex file. Scoring marks are
% presented as a list within the {scorelist} environment. Each
% mark is enlisted using the \scoreitem macro. Headings can be
% defined with the \scoreheading macro. Within the scoresheet
% booklet the {scorelist} environment shall be placed inside the
% {scoresheet} environment that adds the footer and heading required
% by the referee.
%
% -= Snippet (rulebook.tex) =-
%     \newpage%
%     \input{my_score_sheet.tex}
% -= End Snippet =-
%
% -= Snippet (score_sheets.tex) =-
%     \begin[options]{scoresheet}
%     \input{my_score_sheet.tex}
%     \end{scoresheet}
% -= End Snippet =-
%
% -= Snippet (my_score_sheet.tex) =-
%     \begin[options]{scorelist}
%       \scoreheading{Main goal}
%       \scoreitem[multiplier]{score}{Description}
%       % These do not contribute to automatic scoring calculation
%       \scorebonus[multiplier]{score}{Description}
%       \scorepenalty[multiplier]{score}{Description}
%     \end{scorelist}
% -= End Snippet =-
%
%
% scorelist options:
% The scorelist environment supports the following comma-separated
% optional arguments:
%   - score              Integer. Sets the test total score to an
%                        arbitrary value (disables autocalc)
%   - attempts           Integer. Number of attempts for the
%                        scoresheet (default is \global\attempts)
%   - continue           Not implemented
%   - datarecording      Boolean. Toggles the "Data Recording"
%                        item under Special penalties and standard
%                        bonuses
%   - datarecordingpc    Integer. Percentage for the data
%                        recording bonus
%   - datarecordingbonus Integer. Arbitrary value for the data
%                        recording bonus
%   - outstanding        Boolean. Toggles the "Outstanding
%                        Performance" item under Special penalties
%                        and standard bonuses
%   - outstandingpc      Integer. Percentage for the
%                        outstanding performance bonus
%   - outstandingbonus   Integer. Arbitrary value for the
%                        outstanding performance bonus
%   - startbutton        Boolean. Toggles the "Using start button"
%                        item under Special penalties and standard
%                        bonuses.
%   - startbuttonpenalty Integer. Arbitrary value for the Using
%                        start button penalty.
%   - firstcolcaption    String. Caption for the first column of
%                        the scoresheet (default="Action")
%   - secondcolcaption   String. Caption for the second column of
%                        the scoresheet (default="Score")
%   - singletrycc        String. Caption of the first column for
%                        referee scoring when attempts=1
%   - firsttrycc         String. Caption of the first column for
%                        referee scoring when attempts=2
%   - secondtrycc        String. Caption of the second column for
%                        referee scoring when attempts=2
%
%
%
% scoreitem, scorebonus, and scorepenalty arguments
%   #1 multiplier   A number indicating how many times the mark
%                   can be scored. It is printed at the left of
%                   the score followed by the \times symbol.
%   #2 score        Scoring points. Printed at the right of the
%                   description
%   #3 description  A description for the score mark
%
%
%
%
%
%
%
%
%
%
%
%
%
%
%
%
%
%
%
%
%
%
%
%
%
%
%
%
%
%
%
%
%
%
%
%
%
%
%
%
%
%
%
%
%
%
%
%
%
%
% %%% %%%%%%%%%%%%%%%%%%%%%%%%%%%%%%%%%%%%%%%%%%%%%%%%%%%%%%%%%%%%%%
%                                                                  %
% FROM HERE ON, THERE IS NOTHING TO CHANGE                         %
%                                                                  %
% %%% %%%%%%%%%%%%%%%%%%%%%%%%%%%%%%%%%%%%%%%%%%%%%%%%%%%%%%%%%%%%%%

%%% Counters / temp. variables %%%%%%%%%%%%%%%%%
\newcounter{currTestScore}
\newcounter{currTestScoreTotal}
\newcounter{currTestScoreTotalWithoutBonus}
\newcounter{currOutstandingBonus}
\newcounter{currDataRecordingBonus}


% set \continueAvailable to true for CONTINUE sections
\newcommand{\continueAvailable}{true}


% name of the current test, is set automatically in the rulebook
\newcommand{\currentTest}{}

% (internal) if-clause shortcut to switch between short rulebook version and full score sheet for referees
\newcommand{\ifShortScoresheet}[2]{%
	\ifthenelse{ \equal{\shortScoresheet}{true} }{#1}{#2}%
}

% (internal) draws the scoresheet line for score handwritting
\newcommand{\scoreline}[1][0.08]{\rule{#1\linewidth}{.2pt}}

% (internal) returns absolute value of argument
\newcommand{\absval}[1]{\ifnum#1<0 -\fi#1}
















% %%% %%%%%%%%%%%%%%%%%%%%%%%%%%%%%%%%%%%%%%%%%%%%%%%%%%%%%%%%%%%%%%
%                                                                  %
% ENVIRONMENT: scoresheet                                          %
% Scoresheet page layout                                           %
%                                                                  %
% %%% %%%%%%%%%%%%%%%%%%%%%%%%%%%%%%%%%%%%%%%%%%%%%%%%%%%%%%%%%%%%%%

\newenvironment{scoresheet}{%
% \begin{scoresheet}
	\newpage%
	%
	% Test, team, and referee info
	%
	\begin{minipage}[t]{0.85\textwidth}%
		\vspace{0pt}%
		{\huge \textbf{Score Sheet} }%
		\vspace{2 em}%

		\begin{tabular}{ @{} l l l}
			\textbf{Test:} & \currentTest \\[.9 em]%
			\textbf{Team name:} & \scoreline[0.6]\\[.9 em]%
			\textbf{Referee name:} & \scoreline[0.6]\\[.9 em]%
		\end{tabular}%
		\vspace{0.5 em}%

	\end{minipage}
	\hfill
	%
	% @Home Logo
	%
	\begin{minipage}[t]{0.15\textwidth}%
		\vspace{0pt}%
		\includegraphics[width=\textwidth]{images/logo_RoboCupAtHome.jpg}%
	\end{minipage}\\%
}{
% \end{scoresheet}
	\vspace{0.5 em}%
	\textbf{Remarks:}%

	%
	% Signatures of referee / team leader %%%%%%%%%%%%
	%
	\vfill
	\begin{tabular*}{\linewidth}{@{} @{\extracolsep{\fill}} l l l @{}}
		\scoreline[0.25] \hspace{0.05\linewidth}%
			& \scoreline[0.25] \hspace{0.05\linewidth}%
			& \scoreline[0.25]%
		\\
		\textit{Date \& time}%
			& \textit{Referee} %
			& \textit{Team leader}%
	\end{tabular*}

	\newpage
}














% %%% %%%%%%%%%%%%%%%%%%%%%%%%%%%%%%%%%%%%%%%%%%%%%%%%%%%%%%%%%%%%%%
%                                                                  %
% ENVIRONMENT: scorelist                                           %
% Score list table                                                 %
%                                                                  %
% %%% %%%%%%%%%%%%%%%%%%%%%%%%%%%%%%%%%%%%%%%%%%%%%%%%%%%%%%%%%%%%%%

\usepackage{pgfkeys}
\pgfkeys{
	/scorelist/.is family, /scorelist,
	default/.style={
		attempts = \attempts,
		continue = true,
		datarecording = \dataRecordingBonus,
		datarecordingpc = \dataRecordingBonusPercentage,
		datarecordingbonus = 0,
		outstanding = \outstandingPerformanceBonus,
		outstandingpc = \outstandingPerformanceBonusPercentage,
		outstandingbonus = 0,
		startbutton = \startbuttonpenalized,
		startbuttonpenalty = \startbuttonpenalty,
		firstcolcaption = \firstColumnCaption,
		secondcolcaption = \secondColumnCaption,
		singletrycc = \singleTryColumnCaption,
		firsttrycc = \firstTryColumnCaption,
		secondtrycc = \secondTryColumnCaption,
	},
	attempts/.estore in = \scorelistAttempts,
	continue/.estore in = \scorelistContinue,
	datarecording/.estore in = \scorelistDataRecording,
	datarecordingpc/.estore in = \scorelistDataRecordingPercentage,
	datarecordingbonus/.estore in = \scorelistDataRecordingBonus,
	outstanding/.estore in = \scorelistOutstanding,
	outstandingpc/.estore in = \scorelistOutstandingPercentage,
	outstandingbonus/.estore in = \scorelistOutstandingBonus,
	startbutton/.estore in = \scorelistStartButton,
	startbuttonpenalty/.estore in = \scorelistStartButtonPenalty,
	firstcolcaption/.estore in = \scorelistFirstColCaption,
	secondcolcaption/.estore in = \scorelistSecondColCaption,
	singletrycc/.estore in = \scorelistSingleTryCC,
	firsttrycc/.estore in = \scorelistFirstTryCC,
	secondtrycc/.estore in = \scorelistSecondTryCC,
}

\makeatletter%
\NewEnviron{scorelist}[1][]{
% \begin{scorelist}
	%%%%%%%%%%%%%%%%%%%%%%%%%%%%%%%%%%%%%%%%%%%%%%%%%%%%%%%%%%%%%%%
	% read options
	%%%%%%%%%%%%%%%%%%%%%%%%%%%%%%%%%%%%%%%%%%%%%%%%%%%%%%%%%%%%%%%
	\pgfkeys{/scorelist, default, #1}%

	%%%%%%%%%%%%%%%%%%%%%%%%%%%%%%%%%%%%%%%%%%%%%%%%%%%%%%%%%%%%%%%
	% init variables %%%%%%%%%%%%%%%%%%%%%%%%%%%%%%%%%%%%%%%%%%%%%%
	%%%%%%%%%%%%%%%%%%%%%%%%%%%%%%%%%%%%%%%%%%%%%%%%%%%%%%%%%%%%%%%
	\setcounter{currTestScore}{0}
	\setcounter{currOutstandingBonus}{\scorelistOutstandingBonus}
	\setcounter{currDataRecordingBonus}{\scorelistDataRecordingBonus}

	%%%%%%%%%%%%%%%%%%%%%%%%%%%%%%%%%%%%%%%%%%%%%%%%%%%%%%%%%%%%%%%
	% environment commands %%%%%%%%%%%%%%%%%%%%%%%%%%%%%%%%%%%%%%%%
	%%%%%%%%%%%%%%%%%%%%%%%%%%%%%%%%%%%%%%%%%%%%%%%%%%%%%%%%%%%%%%%

	% heading %%%%%%%%%%%%%%%%%%%%%%%%%%%%%%%%%%%%%%%%%%%%%%%%%%%%%
	\newcommand{\scoreheading}[1]{%
		\ifShortScoresheet{%
			\xdef\@scoreheadingcolspan{2}%
		}{%
			\xdef\@scoreheadingcolspan{\the\numexpr2+\scorelistAttempts\relax}%
		}%
		\phantom{.} \\[-12pt]%
		\multicolumn{\@scoreheadingcolspan}{@{}l}{\textbi{##1}}\\[0pt]%
	}

	\newcommand{\scoreitem}[3][1]{%
		\ifthenelse{ ##2 > 0 }{%
			\addtocounter{currTestScore}{ ##2 * ##1 }%
		}{}%
		%
		\@scoreitem[##1]{##2}{##3}%
	}

	\newcommand{\penaltyitem}[3][1]{%
		\ifthenelse{##2 < 0}{%
			\@scoreitem[##1]{\absval{##2}}{##3}%
		}{%
			\@scoreitem[##1]{\absval{-##2}}{##3}%
		}%
	}

	\newcommand{\bonusitem}[3][1]{%
		\@scoreitem[##1]{##2}{##3}%
	}

	% Alias of \bonusitem
	\newcommand{\scorebonus}[3][1]{\bonusitem[##1][##2][##3]}

	% Alias of \penaltyitem
	\newcommand{\scorepenalty}[3][1]{\penaltyitem[##1][##2][##3]}

	%%%%%%%%%%%%%%%%%%%%%%%%%%%%%%%%%%%%%%%%%%%%%%%%%%%%%%%%%%%%%%%
	% Commands for overriding internal calculations %%%%%%%%%%%%%%%
	%%%%%%%%%%%%%%%%%%%%%%%%%%%%%%%%%%%%%%%%%%%%%%%%%%%%%%%%%%%%%%%

	% set score counter to arbitrary value %%%%%%%%%%%%%%%%%%%%%%%%
	\newcommand{\setTotalScore}[1]%
	{%
		\setcounter{currTestScore}{##1}%
	}

	% set outstanding bonus counter to arbitrary value %%%%%%%%%%%%
	\newcommand{\setOutstandingBonus}[1]%
	{%
		\setcounter{currOutstandingBonus}{##1}%
	}

	%%%%%%%%%%%%%%%%%%%%%%%%%%%%%%%%%%%%%%%%%%%%%%%%%%%%%%%%%%%%%%%
	% environment internal commands %%%%%%%%%%%%%%%%%%%%%%%%%%%%%%%
	%%%%%%%%%%%%%%%%%%%%%%%%%%%%%%%%%%%%%%%%%%%%%%%%%%%%%%%%%%%%%%%

	% table entry %%%%%%%%%%%%%%%%%%%%%%%%%%%%%%%%%%%%%%%%%%%%%%%%%
	\newcommand{\@scoreitem}[3][1]{%
		##3\vspace{0.1em} &%
		\textit{%
			\ifthenelse{ \equal{##2}{0} }{~}{% else
				\ifthenelse{ \equal{##1}{1} }{}{##1$\times$}%
			##2}%
		}%
		\ifShortScoresheet{}{&\attemptScoreLines{\scorelistAttempts}}\\[0pt]%
	}

	% [INTERNAL] writes down the line for the final total score %%%
	\newcommand{\scoreTotal}{%
		\\%
		\textbf{Total score~}%
		\ifShortScoresheet{%
			(excluding penalties and standard bonuses) &%
			\textit{\thecurrTestScoreTotalWithoutBonus}%
		}{%
			&\textit{\thecurrTestScoreTotal}
			&\multicolumn{\scorelistAttempts}{c}{\scoreline[0.20]}%
		}\\[0pt]%
	}

	% [INTERNAL] writes down the lines for the total score per try
	\newcommand{\scorePerTry}{%
		\\%
		\ifShortScoresheet{}{%
			\ifthenelse{ \attempts > 1 }{%
				\textbi{Score per try} &%
				\textit{\thecurrTestScoreTotalWithoutBonus} &%
				\attemptScoreLines{\scorelistAttempts}%
				\\[0pt]%
			}{}%
		}%
	}

	% [INTERNAL] draws a line for referee scoring %%%%%%%%%%%%%%%%%
	\gdef\@marklinewidth{0.06}
	\newcommand{\markline}{\rule{\@marklinewidth\linewidth}{.2pt}}

	% [INTERNAL] draws all the line for referee scoring %%%%%%%%%%%
	\newcommand{\attemptScoreLines}[1]{%
		\protected@xdef\@scorelines{\markline}%
		\ifthenelse{##1 > 1}{%
			\foreach \i in {2,...,##1}{%
				\protected@xdef\@scorelines{\@scorelines & \markline}%
			}%
		}{}%
		\@scorelines%
	}

	% [INTERNAL] writes down the headings for referee scoring %%%%%
	\newcommand{\attemptHeadings}[1]{%
		\ifthenelse{\equal{##1}{1}}{%
			\gdef\@attemptheadings{\textbf{\scorelistSingleTryCC}}%
			\gdef\@marklinewidth{0.1}%
		}{}%
		\ifthenelse{\equal{##1}{2}}{%
			\gdef\@attemptheadings{\textbf{\scorelistFirstTryCC} & \textbf{\scorelistSecondTryCC}}%
			\gdef\@marklinewidth{0.08}%
		}{}%
		\ifthenelse{##1 > 2}{
			\protected@xdef\@attemptheadings{%
				\textbf{\small$1^{st}$~try} &%
				\textbf{\small$2^{nd}$~try} &%
				\textbf{\small$3^{rd}$~try}}%
		}{}%
		\ifthenelse{##1 > 3}{
			\foreach \i in {4,...,##1}{% chktex11
				\protected@xdef\@attemptheadings{%
					\@attemptheadings &%
					\textbf{\small$\i^{th}$~try}%
				}%
			}%
			\gdef\@marklinewidth{0.06}%
		}{}%
		\@attemptheadings%
	}

	%%%%%%%%%%%%%%%%%%%%%%%%%%%%%%%%%%%%%%%%%%%%%%%%%%%%%%%%%%%%%%%
	% setup table %%%%%%%%%%%%%%%%%%%%%%%%%%%%%%%%%%%%%%%%%%%%%%%%%
	%%%%%%%%%%%%%%%%%%%%%%%%%%%%%%%%%%%%%%%%%%%%%%%%%%%%%%%%%%%%%%%
	\vspace{0.8 em}%
	\noindent%
	\begin{tabularx}{\textwidth}{ @{}X @{}r *{\scorelistAttempts}{c}}
		\textbf{\scorelistFirstColCaption} &%
		\textbf{\scorelistSecondColCaption}%
		\ifShortScoresheet{}{&\attemptHeadings{\scorelistAttempts}}
	\\\hline



\BODY



	%%%%%%%%%%%%%%%%%%%%%%%%%%%%%%%%%%%%%%%%%%%%%%%%%%%%%%%%%%%%%%%
	% calculate max. score, and bonuses %%%%%%%%%%%%%%%%%%%%%%%%%%%
	%%%%%%%%%%%%%%%%%%%%%%%%%%%%%%%%%%%%%%%%%%%%%%%%%%%%%%%%%%%%%%%
	% base total score (accumulative) %%%%%%%%%%%%%%%%%%%%%%%%%%%%%
	\setcounter{currTestScoreTotal}{\thecurrTestScore}
	% outstanding performance bonus %%%%%%%%%%%%%%%%%%%%%%%%%%%%%%%
	\ifthenelse{\equal{\scorelistOutstanding}{true}}{%
		\ifthenelse{ \equal{\thecurrOutstandingBonus}{0} }{%
			\setcounter{currOutstandingBonus}{ \thecurrTestScore*\scorelistOutstandingPercentage/100 }
		}{}%
		\setcounter{currTestScoreTotal}{%
			\thecurrTestScoreTotal + \thecurrOutstandingBonus}%
	}{}%
	% data recording bonus %%%%%%%%%%%%%%%%%%%%%%%%%%%%%%%%%%%%%%%%
	\ifthenelse{\equal{\scorelistDataRecording}{true}}{%
		\ifthenelse{ \equal{\thecurrDataRecordingBonus}{0} }{%
			\setcounter{currDataRecordingBonus}{ \thecurrTestScore*\scorelistDataRecordingPercentage/100 }%
		}{}%
		\setcounter{currTestScoreTotal}{%
			\thecurrTestScoreTotal + \thecurrOutstandingBonus}%
	}{}%
	\setcounter{currTestScoreTotalWithoutBonus}{ \thecurrTestScore }

	%%%%%%%%%%%%%%%%%%%%%%%%%%%%%%%%%%%%%%%%%%%%%%%%%%%%%%%%%%%%%%%
	% Special penalties & bonuses %%%%%%%%%%%%%%%%%%%%%%%%%%%%%%%%%
	%%%%%%%%%%%%%%%%%%%%%%%%%%%%%%%%%%%%%%%%%%%%%%%%%%%%%%%%%%%%%%%
	\scoreheading{Special penalties \& standard bonuses}

	% not showing up penalty %%%%%%%%%%%%%%%%%%%%%%%%%%%%%%%%%%%%%%
	\penaltyitem{\notattendingpenalty}{Not attending \ifShortScoresheet{(see sec.~\ref{rule:not_attending})}{}}

	% require signal for door opening %%%%%%%%%%%%%%%%%%%%%%%%%%%%%
	\ifthenelse{ \equal{\scorelistStartButton}{true} }{
	  \penaltyitem{\scorelistStartButtonPenalty}{Using start button \ifShortScoresheet{(see sec.~\ref{rule:start_button})}{}}
	}{}

	% data recording bonus %%%%%%%%%%%%%%%%%%%%%%%%%%%%%%%%%%%%%%%%
	\ifthenelse{%
		\thecurrDataRecordingBonus>0 \AND %
		\equal{\scorelistDataRecording}{true}%
	}{%
		\bonusitem{\thecurrDataRecordingBonus}{Contributing with recorded data ($\frac{\sum gathered~points}{max~points} \times$) \ifShortScoresheet{(see sec.~\ref{rule:datarecording})}{}}%
	}{}%

	% outstanding performance bonus %%%%%%%%%%%%%%%%%%%%%%%%%%%%%%%
	\ifthenelse{%
		\value{currOutstandingBonus}>0 \AND %
		\equal{\scorelistOutstanding}{true}%
	}{%
		\bonusitem{\thecurrOutstandingBonus}{Outstanding performance~\ifShortScoresheet{(see sec.~\ref{rule:outstanding_performance})}{}}%
	}{}%
	%
	% Total score %%%%%%%%%%%%%%%%%%%%%%%%%%%%%%%%%%%%%%%%%%%%%%%%%
	\\[-1em]\hline
	\scorePerTry
	\scoreTotal

	\end{tabularx}
}
\makeatother%

% Local Variables:
% TeX-master: "../Rulebook"
% End:
%

\input{./setup/macros_open_demonstrations.tex}
\input{./setup/macros_leagues.tex}

\newcommand{\rulebookVersion}{\STATE\ version for RoboCup \YEAR\xspace(\VERSION)}

\def\RoboCup{{\textsc{RoboCup}}}
\def\Robocup{{\textsc{RoboCup}}}
\def\robocup{{\textsc{RoboCup}}}
\def\AtHome{{\textsc{RoboCup@Home}}}
\def\MSL{\textsc{Middle Size League}}
\def\MS{{\textsc{Middle Size}}}
\def\SSL{\textsc{Soccer Simulation League}}
\def\SS{\textsc{Soccer Simulation}}

\def\TC{technical committee (TC)}
\def\OC{organizing committee (OC)}

\newcommand{\textbi}[1]{\textbf{\textit{#1}}}
\renewcommand{\labelenumi}{\arabic{enumi}.}
\renewcommand{\labelenumii}{\labelenumi\arabic{enumii}.}
\renewcommand{\labelenumiii}{\labelenumii.\arabic{enumiii}.}

\newcommand{\testtocentry}[1]{%
	\nameref{#1}\dotfill\pageref{#1}\\[0.2\baselineskip]%
}

%% %%%%%%%%%%%%%%%%%%%%%%%%%%%%%%%%%%%%%%%%%%%%%%%%%%%%%%%%% %%
%%                    Developement-Tools                     %%
%% %%%%%%%%%%%%%%%%%%%%%%%%%%%%%%%%%%%%%%%%%%%%%%%%%%%%%%%%% %%

%% %%%%%%%%%%%%%%%%%%%%%%%%%%%%%%%%%%%%%%%
\newcommand{\tbc}[1]{\textbf{\it\color{red}{t.b.c. ...}#1\color{black}}}
\newcommand{\todo}[1]{\textbf{\it\color{red}{todo: }#1\color{black}}}
\newcommand{\TODO}[1]{\textbf{\it\color{red}{TODO:\\}#1\color{black}}}
\newcommand{\chk}[1]{\textbf{\color{red}#1\color{black}}}

\newcommand{\reworkon}{\marginpar{\raggedright\color{red}{$\downarrow$rework}\color{black}}}
\newcommand{\reworkoff}{\marginpar{\raggedright\color{red}{$\uparrow$rework}\color{black}}}

%% %%%%%%%%%%%%%%%%%%%%%%%%%%%%%%%%%%%%%%%
%%  site notes/margin notes
\def\note#1{\marginpar{\raggedright\tiny#1}}
\def\mpar#1{\marginpar{\raggedright\tiny#1}}
\def\rand#1{\marginpar{\raggedright\tiny#1}}
\setlength{\marginparwidth}{2cm}

\newcommand{\refsec}[1]{Section~\ref{#1}}
\newcommand{\reftab}[1]{Table~\ref{#1}}
\newcommand{\reffig}[1]{Figure~\ref{#1}}

%% %%%%%%%%%%%%%%%%%%%%%%%%%%%%%%%%%%%%%%%
%% side-annotation-macros for easy lookup
% \newcommand{\awardmark}{\marginpar{\centering\includegraphics[width=.34cm]{images/icon_award.pdf}}}
% \newcommand{\refmark}{\marginpar{\centering\includegraphics[width=.5cm]{images/icon_whistle.pdf}}}
% \newcommand{\referee}[1]{\emph{#1}\marginpar{\centering\includegraphics[width=.5cm]{images/icon_whistle.pdf}}}
% \newcommand{\scoremark}{\marginpar{\centering\includegraphics[width=.34cm]{images/icon_score.pdf}}}
\newcommand{\awardmark}{}
\newcommand{\refmark}{}
\newcommand{\referee}[1]{}
\newcommand{\scoremark}{}
%\newcommand{\scoring}[1]{\emph{#1}\marginpar{\centering\includegraphics[width=.34cm]{images/icon_score.pdf}}}
\newcommand{\scoring}[1]{\emph{#1}}
\newcommand{\timark}{\marginpar{\centering\includegraphics[width=.34cm]{icon_clock.pdf}}}

%\newcommand{\timing}[1]{\emph{#1}\marginpar{\centering\includegraphics[width=.34cm]{images/icon_clock.pdf}}}
\newcommand{\timing}[1]{\emph{#1}}

\def\svnRevision{Unknown} %
\def\svnChangeData{Unknown} %
\def\revnumtmpfile{.temp_ruleook_version}
\def\revdattmpfile{.temp_ruleook_date}
\immediate\write18{git rev-list HEAD | wc -l > \revnumtmpfile}
%\immediate\write18{svnversion . > \revnumtmpfile}
\IfFileExists{\revnumtmpfile}{\def\svnRevision{\input{\revnumtmpfile}\unskip}}{}
\immediate\write18{git log -1 --date=short  | grep 'Date:' | awk '{print $2}'> \revdattmpfile}
%\immediate\write18{svn info | grep 'Last Changed Date:' | awk '{print $4}'> \revdattmpfile}
\IfFileExists{\revdattmpfile}{\def\svnChangeData{\input{\revdattmpfile}\unskip}}{}
% \IfFileExists{\revnumtmpfile}{\immediate\write18{rm -f \revnumtmpfile}}{}
% \IfFileExists{\revdattmpfile}{\immediate\write18{rm -f \revdattmpfile}}{}
\newcommand{\VERSION}{Revision \svnChangeData\_\svnRevision}


% Local Variables:
% TeX-master: "../Rulebook"
% End:

\input{./setup/abbrevix.tex}



\makeindex                                % generate index
\makeabbex                                % generate abbreviations

%%% DOCUMENTINFO %%%%%%%%%%%%%%%%%%%%%%%%%%%%%%%%%%%%%%%%%%%%%%%%%%%%%%%%%%%%%%
\hypersetup{
  pdftitle     = {RoboCup@Home Rules and Regulations},
  pdfsubject   = {RoboCup@Home Rulebook},
  pdfauthor    = {RoboCup@Home Technical Committee},
  pdfkeywords  = {RoboCup, @Home, Rules, Competition},
  colorlinks   = true,
  anchorcolor  = blue,
  linkcolor    = blue,
  urlcolor     = blue,
}

%%% HEADINGS & PAGE STYLE %%%%%%%%%%%%%%%%%%%%%%%%%%%%%%%%%%%%%%%%%%%%%%%%%%%%%
\newcommand{\footline}{RoboCup@Home Rulebook / \rulebookVersion}
\pagestyle{fancy}
\renewcommand{\chaptermark}[1]{\markboth{\chaptername\ \thechapter. \ #1}{}}
\renewcommand{\sectionmark}[1]{\markright{\thesection \ #1}{}\renewcommand{\currentTest}{#1}}
\fancyhf{}
\fancyhead[LE,RO]{\thepage}
\fancyhead[RE]{\sffamily\rightmark}
\fancyhead[LO]{\sffamily\leftmark}
\fancyfoot[C]{\scriptsize \sffamily \footline{}}
\fancypagestyle{plain}{
        \fancyhf{}
        \fancyhead[LE,RO]{\thepage}
        \fancyhead[RE]{\sffamily\rightmark}
        \fancyhead[LO]{\sffamily\leftmark}
        \fancyfoot[C]{\scriptsize \sffamily \footline{}}
		\renewcommand{\headrulewidth}{0.5 pt}
}
\fancypagestyle{empty}{
        \fancyhf{}
        \fancyhead{}
        \fancyfoot[C]{\scriptsize \sffamily \footline{}}
		\renewcommand{\headrulewidth}{0 pt}
}

%\newcommand{\sectionbreak}{\clearpage}
%\newcommand{\subsectionbreak}{\clearpage}


%%%%%%%%%%%%%%%%%%%\renewcommand{%%%%%%%%%%%%%%%%%%%%%%%%%%%%%%%%%%%%%%%%%%%%%%%%%%%%%%%%%%%%
%%%%%%%%%%%%%%%%%%%%%%%%%%%%%%%%%%%%%%%%%%%%%%%%%%%%%%%%%%%%%%%%%%%%%%%%%%%%%%%
%%%%%%%%%%%%%%%%%%%%%%%%%%%%%%%%%%%%%%%%%%%%%%%%%%%%%%%%%%%%%%%%%%%%%%%%%%%%%%%

\begin{document}

\input{./pages/titlepage}

\pagestyle{empty}
%% %%%%%%%%%%%%%%%%%%%%%%%%%%%%%%%%%%%%%%%%%%%%%%%%%%%%%%%%%%%%%%%%%%%%%%%%%%%
%%
%%          $Id: acknowledgments.tex 404 2013-02-15 08:51:20Z sugiura $
%%    author(s): RoboCupAtHome Technical Committee(s)
%%  description: Acknowledgments for the RoboCupAtHome RuleBook
%%
%% %%%%%%%%%%%%%%%%%%%%%%%%%%%%%%%%%%%%%%%%%%%%%%%%%%%%%%%%%%%%%%%%%%%%%%%%%%%



\section*{About This Rulebook}
This is the official rulebook of the \YEAR ~RoboCup@Home competition.
It has been written by the \YEAR ~RoboCup@Home Technical Committee with the special collaboration of (in alphabetical order):
% Mauricio Matamoros,
% and
% Loy van Beek.



\section*{How to Cite This Rulebook}
If you refer to RoboCup@Home and this rulebook in particular, please cite:

Mauricio Matamoros, Caleb Rascon, Justin Hart, Dirk Holz, Kai Chen, and Loy van Beek.
\enquote{Robocup@Home \YEAR: Rule and regulations,}
\url{http://www.robocupathome.org/rules/2019_rulebook.pdf}, \YEAR.

\begin{center}
\begin{minipage}{0.8\textwidth}
	\footnotesize%
	\verbatiminput{citation.bib}
\end{minipage}
\end{center}

\section*{Acknowledgments}
\label{sec:acknowledgments}
We would like to thank the members of the Technical Committee who put up the rules and the Organizing Committee who organizes the competition.

~\\\noindent People that have been working on this rulebook as member of one of the league's committees (in alphabetical order):
\begin{center}
\begin{minipage}{0.8\textwidth}
\begin{multicols}{3}%
\footnotesize
\noindent%

% Kai Chen\\
% Justin Hart\\
% Luca Iocchi\\
% Mauricio Matamoros\\
% Dirk Holz\\
% \columnbreak
% Raphael Memmesheimer\\
% Alexander Moriarty\\
% Caleb Rascon\\
% Sammy Pfeiffer\\
% \columnbreak
% Komei Sugiura\\
% Sven Wachsmuth\\
% Tijn van der Zant\\
\end{multicols}
\end{minipage}
\end{center}

We also like to thank all the people who contributed to the RoboCup@Home league with their feedback and comments.

~\\\noindent People that have been working on this rulebook as member the league (in alphabetical order):
\begin{center}
\begin{minipage}{0.8\textwidth}
\begin{multicols}{2}%
\footnotesize
\noindent%
% Loy van Beek (@LoyVanBeek)\\
% Sebastian Meyer zu Borgsen (@semeyerz)\\
% Matthijs van der Burgh (@MatthijsBurgh)\\
% Remi Fabre (@RemiFabre)\\
% Tarik Kelestemur (@tkelestemur)\\
% \columnbreak%
% Johannes Kummert (@johaq)\\
% Florian Lier (@warp1337)\\
% Hiroyuki Okada (@okadahiroyuki)\\
% @AMR-\\
% @fibonatic\\
\end{multicols}
\end{minipage}
\end{center}


% Local Variables:
% TeX-master: "../Rulebook"
% End:

\clearpage

\input{Changelog}
\clearpage

\pagestyle{empty}
\tableofcontents
\clearpage

\pagestyle{plain}

\input{Introduction}

\chapter{Concepts behind the competition}
\label{chap:concepts}
A set of conceptual key criteria builds the basis for the RoboCup@Home Competitions. These criteria are to be understood as a common agreement on the general concept of the competition. The concrete rules are listed in Chapter \refsec{chap:rules}.

\section{Lean set of rules}
\label{concept:lean_set_of_rules}
To allow for different, general and transmissible approaches in the RoboCup@Home competitions, the rule set should be as lean as possible. Still, to avoid rule discussions during the competition itself, it should be very concrete leaving no room for diverse interpretation.

If, during a competition, there are any discrepancies or multiple interpretations, a decision will be made by the \iaterm{Technical Committee}{TC} and the referees on site.

\paragraph*{Note: } Once the test scoresheet has been signed or the scores has been published, the TC decision is irrevocable.

\section{Autonomy \& Mobility}
\label{concept:autonomy_and_mobility}
All robots participating in the RoboCup@Home competition have to be \emph{autonomous} and \emph{mobile}.

An aim of RoboCup@Home is to foster mobile autonomous service robotics and natural human-robot interaction. As a consequence humans are not allowed to directly (remote) control the robot. This also includes verbally remote controlling the robot.

Furthermore, the specific tasks must not be solved using \emph{open loop control}.

\section{Aiming for applications}
\label{concept:aiming_for_applications}
To foster advance in technology and to keep the competition interesting, the scenario and the tests will steadily increase in complexity. While in the beginning necessary abilities are being tested, tests will focus more and more on real applications with a rising level of uncertainty. Useful, robust, general, cost effective, and applicable solutions are rewarded in RoboCup@Home.

\section{\iterm{Social relevance}}
\label{concept:social_relevance}
The competition and the included tests should produce socially relevant results. The aim is to convince the public about the usefulness of autonomous robotic applications. This should be done by showing applications where robots directly help or assist humans in everyday life situations. Examples are: Personal robot assistant, guide robot for the blind, robot care for elderly people, etc. Such socially relevant results are rewarded in RoboCup@Home.

\section{Scientific value}
\label{concept:scientific_value}
RoboCup@Home should not only show what can be put into practice today, but should also present new approaches, even if they are not yet fully applicable or demand a very special configuration or setup. Therefore high scientific value of an approach is rewarded.

\section{Time constraints}
\label{concept:time_constraints}
Setup time as well as time for the accomplishment of the tests is very limited, to allow for many participating teams and tests, and to foster simple setup procedures.

\section{No standardized scenario}
\label{concept:no_standardized_scenario}
The \iterm{scenario} for the competition should be simple but effective, available world-wide and low in costs. As uncertainty is part of the concept, no standard scenario will be provided in the RoboCup@Home League. One can expect that the scenario will look typical for the country where the games are hosted.

The scenario is something that people encounter in daily life. It can be a home environment, such as a living room and a kitchen, but also an office space, supermarket, restaurant etc. The scenario should change from year to year, as long as the desired tests can still be executed.

Furthermore, tests may take place outside of the scenario, i.e., in an previously unknown environment like, for example, a public space nearby.

\section{Attractiveness}
\label{concept:attractiveness}
The competition should be attractive for the audience and the public. Therefore high attractiveness and originality of an approach should be rewarded.

\section{\iterm{Community}}
\label{concept:community}
Though having to compete against each other during the competition, the members of the RoboCup@Home league are expected to cooperate and exchange knowledge to advance technology together. The \iterm{RoboCup@Home mailing list} can be used to get in contact with other teams and to discuss league specific issues such as rule changes, proposals for new tests, etc.
% Since 2007 there is also the \iterm{RoboCup@Home Wiki} (see \refsec{sec:at_home_wiki}) which serves as a central place to collect information relevant for the @Home league.
Every team is expected to share relevant technical, scientific (and team related) information there and in its \iterm{team description paper} (see \refsec{rule:website_tdp}) through the team's website.

All teams are invited to submit papers on related research to the RoboCup Symposium which accompanies the annual RoboCup World Championship.

\section{Desired abilities}
\label{concept:desired_abilities}
This is a list of the current desired technical abilities which the tests in RoboCup@Home will focus on.

\begin{itemize}
\item Navigation in dynamic environments
\item Fast and easy calibration and setup \\ The ultimate goal is to have a robot up and running out of the box.
\item Object recognition
\item Object manipulation
\item Detection and Recognition of Humans
\item Natural human-robot interaction
\item Speech recognition
\item Gesture recognition
\item Robot applications \\ RoboCup@Home is aiming for applications of robots in daily life.
\item Ambient intelligence, e.g., communicating with surrounding devices, getting information from the internet etc.
\end{itemize}


% Local Variables:
% TeX-master: "Rulebook"
% End:


%% %%%%%%%%%%%%%%%%%%%%%%%%%%%%%%%%%%%%%%%%%%%%%%%%%%%%%%%%%%%%%%%%%%%%%%%%%%%
%%
%%          $Id: general_rules.tex 420 2013-04-08 15:30:35Z holz $
%%    author(s): RoboCupAtHome Technical Committee(s)
%%  description: description of the GENERAL RULES
%%
%% %%%%%%%%%%%%%%%%%%%%%%%%%%%%%%%%%%%%%%%%%%%%%%%%%%%%%%%%%%%%%%%%%%%%%%%%%%%
\chapter{General Rules \& Regulations}
\label{chap:rules}

These are the general rules and regulations for the competition in the \RoboCup\AtHome{} league.
They apply to every test unless a test description differs, in which case it overrides the general rule.

\input{general_rules/TeamRegistration}


%%%%%%%%%%%%%%%%%%%%%%%%%%%%%%%%%%%%%%%%%%%%%%%%%%%%%%%%%
\section{Scenario}
\label{sec:scenario}

The tests take place in the \iterm{RoboCup@Home arena}. Nonetheless, some tests can take place outside the arena, in a previously unknown public place. Rules in this section are related to the \iterm{RoboCup@Home arena} and its contents.

\subsection{RoboCup@Home arena}
The \iterm{RoboCup@Home arena} is a realistic home setting (apartment) consisting of inter-connected rooms.
The minimal configuration consists of
\begin{itemize}
	\item bedroom,
	\item dining room,
	\item living room, and
	\item kitchen.
\end{itemize}
Depending on the Local Organization, there may be multiple apartments which may be different to each other.
Robot must be prepared to perform any task in any arena, not the same arena every time.

The arena is decorated and dressed to resemble a typical apartment in the hosting country, including all necessities and decorations one can find in a normal house.
Please do note that what is considered as \enquote{normal} may greatly vary by culture and on the location where the RoboCup event is hosted.
Decorations include, but are not limited to: plants, mirrors, paintings, posters, plates, picture frames, wall clocks, candles with holders, and books.
For a description of objects, please refer to \refsec{rule:scenario_objects}

\subsection{Walls, doors and floor}
\label{rule:scenario_walls}

The indoor home setting will be surrounded by high and low \Term{walls}{Arena walls}.
These walls will be built up using standard fair construction material.

\begin{enumerate}
	\item \textbf{Walls:} Walls have a minimum height of \SI{60}{\centi\meter}. A maximum height is not specified, but must allow the audience to watch the competition.\\
	Walls are fixed and not to be modified during the competition (see~\refsec{rule:scenario_changes}).

	\item \textbf{Doors:} There will be at least two \Term{doors}{Arena doors}, an entrance and an exit, to be used as starting points for the robots (see~\refsec{rule:start_position}).
	% At least one of the entrances will be a door with a handle (not a knob).\
	Inside the arena rooms are connected by doors (at least one).
	All doors have handles, not knobs.
	Doors can be closed at any time, and it is expected that robots be able to open them.

	\item \textbf{Floor:} The floor of the arena as well as the doorways of the arena are even.
	That is, there will be no significant steps or even stairways.
	However, minor unevenness such as carpets, transitions in floor covering between different areas, and minor gaps (especially at doorways) can be expected.

	\item \textbf{Appearance:} Floor and walls are mainly uni-colored but can contain texture, e.g., a carpet on the floor, or a poster or picture on the wall.\\
	Although being unlikely at the moment, transparent elements are also possible.
\end{enumerate}


\subsection{Furniture}
\label{rule:scenario_furniture}
The arena will be equipped with typical objects (furniture) that are not specified in quantity and kind.

The minimal configuration consists of:
\begin{itemize}
	\item a bed,
	\item a couch,
	\item a small table,
	\item a small dinner table with two chairs,
	\item a coat rack or pole,
	\item two trash bins,
	\item an open cupboard or small table with a television and remote control,
	\item a cupboard with drawers, and
	\item a bookcase or shelf with doors and some books inside
\end{itemize}

Likewise the arena's kitchen must have:
\begin{itemize}
	\item a dishwasher,
	\item a microwave,
	\item a sink, and
	\item a refrigerator in the kitchen (with some cans and plastic bottles inside).
\end{itemize}

A typical arena setup is shown in~\reffig{fig:scenario_arena}.

\begin{figure}[tbp]
	\centering
	\subfloat[Typical arena]{\label{fig:scenario_arena}\includegraphics[height=46mm]{images/typical_arena.jpg}} ~
	\subfloat[Typical objects]{\label{fig:scenario_objects}\includegraphics[height=46mm]{images/typical_objects.jpg}}
	\caption{Scenario examples: (a) a typical arena, and (b) typical objects.}
	\label{fig:arena}
\end{figure}


\subsubsection{Cupboard}
The cupboard can be any shelf-like furniture in which objects can be placed.
\begin{itemize}
	\item[\textbf{Doors:}] The cupboard may have doors.
	\item[\textbf{Drawers:}] The cupboard must have at least two drawers betweem 90cm and 120cm from floor level.
	\item[\textbf{Shelves:}] The minimum distance between shelf or layers is 30cm.
\end{itemize}

\subsubsection{Shelf}
A shelf, rack, or bookcase is required in RoboCup@Home.
The shelf can be any shelf-like furniture in which objects can be placed.
\begin{itemize}
	\item[\textbf{Doors:}] The shelf must have at least one door (preferrably a vertical one) covering up to one half of it.
	\item[\textbf{Drawers:}] The shelf must have no drawers.
	\item[\textbf{Shelves:}] The shelf must have 5 shelves or layers between 0.0m and 1.80m from the ground, with a minimum distance of 30cm between shelves or layers.
\end{itemize}

\subsubsection{Fridge}
Fridges must not be smaller than 120m. At least one powered and functioning fridge is required.


\subsection{Changes to the arena}
\label{rule:scenario_changes}

Since the robots should be able to function in the real world the scenario is not fixed and might change without further notice.
\begin{enumerate}
	\item \textbf{Major changes:}
	The arena is meant to be a simulated apartment.
	The furniture might be moved around between tests.
	This includes furniture that is a named location (see~\refsec{rule:scenario_names}).
	As in a normal home, furniture is not very likely to move from one room to another and is unlikely to be moved to the other side of a room.
	However, a couch or table may be rotated, moved to its side etc.
	Walls will stay in place and rooms will not change function.
	Passages might be blocked and cleared.
	One hour before a test slot begins no \iterm{major changes} will be made.
	This time will be shortened in the future.

	\item \textbf{Minor changes:} In contrast to major changes, \iterm{minor changes} like, for instance, slightly moved chairs cannot be avoided and may happen at any time (even during a test).
\end{enumerate}


%%%%%%%%%%%%%%%%%%%%%%%%%%%%%%%%%%%%%%%%%%%%%%%%%%%%%%%%%%%%%%%%%%
%
% Objects section.
%
%%%%%%%%%%%%%%%%%%%%%%%%%%%%%%%%%%%%%%%%%%%%%%%%%%%%%%%%%%%%%%%%%%
\def\NumObjects{30\ }
\def\NumLocations{20\ }
\def\NumNames{20\ }

\subsection{Objects}
\label{rule:scenario_objects}
Some tests in the RoboCup@Home league involve recognizing and manipulating \iterm{objects} (See~\reffig{fig:scenario_objects}).
Most objects are likely to be lightweight and easy to grasp with one hand.

There are three types of objects:

\begin{enumerate}
	\item \textbf{\iterm{Standard objects}:} Objects announced prior to the competition.
	These are \textit{YCB Object and Model Set} IDs 1-10, 19, 21 and 22, representing a selection of food items and household cleaners.

	\item \textbf{\iterm{Known objects}:} Objects announced during the setup days (See~\refsec{chap:setup_and_preparation}).
	There are two kinds of known objects:
	\begin{enumerate}
		\item \textbf{\iterm{Regular objects}:} Objects with no noticeable difference among peers (e.g.~soda can, cereal box, cutlery, etc).
		\item \textbf{\iterm{Alike objects}:} Objects which differ from instance to instance, but are still considered the same by people (e.g.~apple, sandwich, cloth, etc.).
	\end{enumerate}

	\item \textbf{\iterm{Unknown objects}:} Any object that is not standard or known.

During setup days, the TC will provide exemplars of standard and known objects as well as an ontology
describing their official names and designated categories (e.g. an \textit{apple} and a \textit{banana} belong to the \textit{fruits} category).
Each \iterm{object category} has a \iterm{predefined location} (e.g. an \textit{fruits} can be found in the \textit{kitchen table}).

\paragraph*{Important note:} It is not allowed to modify any of the objects provided for training.
Teams are not allowed to keep more than 5 of the objects provided for training at a time and must return them after 1 hour.

\end{enumerate}

\subsection{Minimal Set of Known Objects}
\label{rule:scenario_objects_list}

\begin{itemize}
	\item \textbf{\iterm{Tableware}:} Dish, bowl, cup (or mug), and napkin.
	\item \textbf{\iterm{Cutlery}:} Fork, knife, and spoon.
	\item \textbf{\iterm{Bags}:} Lightweight. With stiff, vertical handles.
	\item \textbf{\iterm{Disks or books}:} A set of 10 discs (LP, CD, DVD, or BluRay) or books, all of the same kind.
	\item \textbf{\iterm{Trays}:} A transport object like a tray or basket. Intended for two-handed manipulation.
	\item \textbf{\iterm{Pourable}:} An object whose content can be poured (e.g.~muesli, cereal, etc.).
	\item \textbf{\iterm{Heavy object}:} Weight between 1.0kg and 1.5kg.
	\item \textbf{\iterm{Tiny object}:} A lightweight object with no bigger than 5cm (e.g.~paper, teabag, pen).
	\item \textbf{\iterm{Fragile object}:} An easy-to-break object, (e.g.~chocolate egg).
	\item \textbf{\iterm{Amorphous object}:} An flexible object that may take an infinite number of shapes (e.g.~cloth, magnetic puzzle, etc.).
	\item \textbf{\iterm{Garbage bag}:} A tie-able garbage bag.
\end{itemize}


\begin{figure}[H]
	\centering
	\subfloat[Bright-colored paper bags]{
		\label{fig:scenario_container_bag}\includegraphics[width=0.33\textwidth]{images/container_paper_bag.png}}~
	\subfloat[Cereal bowls]{
		\label{fig:scenario_container_bowl}\includegraphics[width=0.33\textwidth]{images/container_bowl.png}}~
	\subfloat[Serving tray]{
		\label{fig:scenario_container_tray}\includegraphics[width=0.33\textwidth]{images/container_tray.png}}
	\caption{Examples of possible known objects}
	\label{fig:scenario_containers}
\end{figure}

\subsection{Attributes of Objects}
\label{rule:scenario_objects_attributes}
During the competition, objects can be referenced based on their category \iterm{object category}, physical attributes, or a combination of both.
Attributes that may be used are:
\begin{itemize}
	\item Color (e.g. red, blue).
	\item Pattern (e.g. black with white dots)
	\item Relative estimated size (e.g. smallest, largest, big one).
	\item Relative estimated weight (e.g. lightest, heaviest).
	\item Relative position (e.g. left of, right most).
	\item Object description (e.g. is fragile, is container, can be poured, requires two hands).
\end{itemize}

\noindent\textbf{Remark:} Measurements are common sense estimates.
It is OK for robots to consider similar objects to be about the same size or weight.

%%%%%%%%%%%%%%%%%%%%%%%%%%%%%%%%%%%%%%%%%%%%%%%%%%%%%%%%%%%%%%%%%%
%
% Predefined locations section.
%
%%%%%%%%%%%%%%%%%%%%%%%%%%%%%%%%%%%%%%%%%%%%%%%%%%%%%%%%%%%%%%%%%%

\subsection{Predefined rooms and locations}
\label{rule:scenario_locations}
Some tests in the RoboCup@Home league involve \iterm{predefined locations} where people or objects can be found.
The TC will compile a list of predefined locations that may include furniture (e.g. bookshelf), decorations (e.g. plant, mirror), and doors.
Each \iterm{predefined location} has assigned a \iterm{location class} (e.g. an \textit{coach} and a \textit{arm chair} belong to the \textit{seat} class).
Room names, predefined locations, and location classes are announced during setup days (See~\refsec{chap:setup_and_preparation}).



\subsection{Predefined (person) names}\label{rule:scenario_names}
Some tests in the RoboCup@Home league involve memorizing a person name.
All people in the arena has an assigned \iterm{predefined name}.
The TC will compile a list of \NumNames \iterm{predefined names}.
The names are \SI{25}{\percent} male, \SI{25}{\percent} female, and \SI{50}{\percent} gender-neutral, taken from the list of most common used names in the United States.
Predefined names are announced during setup days (See~\refsec{chap:setup_and_preparation}).


\subsection{Wireless network}
\label{rule:scenario_wifi}

For wireless communication, an \iterm{arena network} is provided. The actual infrastructure depends on the local organization.
The organizers do NOT guarantee reliability and performance of wireless communication.
Teams required to start must do so regardless the availability of the network infrastructure.

The following rules apply:

\begin{itemize}
	\item Only the \iterm{arena network} can be used during tests.
	\item During the competitions, only the active team is allowed to use the \iterm{arena network}.
	\item The \iterm{arena network} provides one Virtual Local Area Networks (VLANs) per team.
	\item Each VLAN is most likely to have its own SSID/password.
	\item VLAN traffic is separated from any other team, routed to the team's network cable (team area).
	\item Each VLAN is also connected to the Internet.
\end{itemize}

\indent\textbf{Remark:} Teams broadcasting unauthorized (aka rogue) wireless networks will be disqualified from the competition, and have their devices confiscated by the OC.
This includes smartphones and concealed SSIDs.
It is advised to verify your devices.


% Local Variables:
% TeX-master: "../Rulebook"
% End:


\section{Simulation Platform}
\label{sec:rules:simulation}

Because of COVID-19, this year scenario will be held in a virtual environment.
\subsection{Simulation Software}
\label{sec:rules:simulationsoftware}
The \TC{} has decided to use \href{http://gazebosim.org/}{Gazebo} as the simulation software.

%%%%%%%%%%%%%%%%%%%%%%%%%%%%%%%%%%%%%%%%%%%%%%%%%%%%%%%%%
\section{Robots}
\label{rule:robots}

\subsection{Number of robots}
\label{rule:robots_number}

\begin{enumerate}
	\item \textbf{Registration:} The maximum \term{number of robots} per team that can be registered for the competitions is \emph{two} (2).
	\item \textbf{Regular Tests:} Only one robot is allowed per test. For different tests different robots can be used.
	\item \textbf{Open Demonstrations:} In the \iterm{Open Challenge} and the \iterm{Finals} both robots can be used simultaneously.
\end{enumerate}

\subsection{Appearance and safety}
\label{rule:robot_appearance}

Robots should have a nice product-like appearance, be safe to operate \& be around and should not annoy its human users. The following rules apply to all robots and are part of the \iterm{Robot Inspection} test (see \refsec{sec:robot_inspection}). 
\begin{enumerate}
	\item \textbf{Cover:} The robot's internal hardware (electronics and cables) should be covered in an appealing way. The use of (visible) duct tape is strictly prohibited.
	\item \textbf{Loose cables:} There may not be any loose cables hanging out of the robot. 
	\item \textbf{Safety:} The robot may not have sharp edges or other things that could severe people.
	\item \textbf{Annoyance:} The robot should not permanently make loud noises or use blinding lights.
	\item \textbf{Marks:} The robot may not exhibit any kind of artificial marks or patterns.
	\item \textbf{Driving:} To be safe, the robots should be careful when driving in a direction it cannot sense, for example. 
\end{enumerate}

\subsection{Standard Platform Leagues}
RoboCup@Home features two Standard Platform Leagues adhering to the rules listed above.

\subsubsection{Modifications}
\label{rule:spl-mods}
Standardized platforms allow teams to compete in equality of conditions by eliminating all hardware-dependent variables.
Therefore, modifications and alterations to the robots are strictly forbidden; including, but not limited to attaching, connecting, plugging, gluing, and taping components into and onto the robot, as well as modifying or altering the robot structure.
Voiding this rule leads to immediate disqualification from the competition and penalty for the team (see~\refsec{rule:extraordinary_penalties}).

During the \iterm{Robot Inspection} test (see~\refsec{sec:robot_inspection}), the TC will verify that the robot is in proper state for the competition; presenting no alterations and a neat condition.
EC and TC members may request re-inspection of a SPL robot at any time during the competition.

\textbf{Clothing, coloring, and stickers:} Robots are allowed to \enquote{wear} clothes, as well as have stickers (e.g., a sticker exhibiting the logo of an sponsor).
Painting the robot with another color or design is also allowed. 
However, artificial markers (e.g. bar codes, QR codes, OpenCV markers) are strictly forbidden. 
Teams should contact the robot provider before altering the robot's appearance.

% \subsubsection{Domestic Standard Platform League}
% The characteristics of the Toyota Human Support Robot are detailed below.

% \begin{itemize}
	% \item Aimed at human support tasks, elderly care et cetera
	% \item Omni-directional base, maximum speed 0.8km/h
	% \item 1 arm with multifunctional gripper through a vacuum pad. The wrist is equipped with a force-torque sensor. Capable of lifting 1.2kg.
	% \item RGB-D, stereo cameras and wide-angle camera
	% \item Display mounted in head, separate tablet interface
	% \item Access to cloud-based services
	% \item Equipped with a microphone array
	% \item Gravity compensated arm
	% \item Height-adjusting torso
% \end{itemize}
%
% \subsubsection{Social Standard Platform League}
% The characteristics of the Softbank Robotics/Aldebaran Pepper are detailed below.
%
% \begin{itemize}
	% \item Aimed at social interaction, public environments, explainable artificial intelligence
	% \item Omni-directional base, maximum speed 3km/h
	% \item 2 arms mostly intended for social gesturing.
	% \item 3D and 2 HD cameras
	% \item Equipped with a built-in tablet
	% \item Access to cloud-based services
	% \item Equipped with a 4-microphone array in the head
	% \item Emotion recognition by voice and images
	% \item Emotion engine to adapt it's attitude
% \end{itemize}


\subsection{Open Platform League}
\label{sec:rules:robotappearance_opl}
Robots competing in the \OPL{} must comply with security specifications in order to avoid causing any harm while operating.

\subsubsection{Size and Weight}
\label{sec:rules:robotappearance_opl:size}

\begin{itemize}
	\item \textbf{Dimensions:} The dimensions of a robot should not exceed the limits of an average door (\SI{200}{\centi\meter} by \SI{70}{\centi\meter}). The \TC{} may allow the qualification and registration of larger robots, but it cannot be guaranteed that the robots can actually enter the arena. In doubt, contact the \LOC{}.
\end{itemize}


\noindent\textbf{Note:} All robot requirements will be tested during \RobotInspection{} (see~\ref{sec:setupdays:inspection}).



% Local Variables:
% TeX-master: "../Rulebook"
% End:


% %% %%%%%%%%%%%%%%%%%%%%%%%%%%%%%%%%%%%%%%%%%%%%%%%%%%%%%%%%%
%
% External Devices
%
% %% %%%%%%%%%%%%%%%%%%%%%%%%%%%%%%%%%%%%%%%%%%%%%%%%%%%%%%%%%
\section{External devices}
\label{rule:robot_external_devices}
Everything which is not part of the robot is considered an \iterm{external device}.
All external devices must be authorized by the \iaterm{Technical Committee}{TC} during the \iterm{Robot Inspection} test (see~\refsec{sec:robot_inspection}).
The \iaterm{Technical Committee}{TC} specifies whether an external device can be used freely, under referee supervision, and its impact on scoring.
In general, external devices must be removed quickly after the test.

\noindent \textbf{Remark:} The use of \iterm{wireless devices} is strictly prohibited. \iterm{External microphones}, hand microphones, and headsets are not allowed in OPL and it use is discouraged in DSPL and SSPL.

\subsection{On-site external computing}
Computing resources that are not physical attached to the robot are considered \iterm{external computing resources}.
The use of up to 5 external computing resources is allowed, but only through the arena network (see \refsec{rule:scenario_wifi}) and with the previous approval of the \iaterm{Technical Committee}{TC}.
Teams must announce the use of any external computing resource at least 1 month before the competition to the \iaterm{Technical Committee}{TC}.

External Computing Devices must be placed in the \iaterm{\textbf{E}xternal \textbf{C}omputing \textbf{R}esource \textbf{A}rea}{ECRA} which is announced by the \iaterm{Technical Committee}{TC} during setup days.
A switch connected to the arena wireless network will be available to teams in the ECRA.
It is strictly forbidden to connect any kind of device or peripheral (e.g.~screens, mouses, keyboards, etc.) to the computers in the ECRA during the competition.

A maximum of two laptops and two people from different teams is allowed at any time in the ECRA.
Teams using laptops as External Computing Devices must remove the device immediately after the test.
Once a test has started, all people must stay at least 1 meter from the ECRA.
Interacting with computers in the ECRA after the Referee has given the start signal will cause the immediate disqualification of the team.

\noindent \textbf{Remark:} Robot operation must be able to operate safely when \iterm{external computing resources} are unavailable.



% On-line devices
\subsection{On-line external computing}
\label{rule:robot_external_computing_online}
Robots are allowed to use \enquote{Cloud services}, \enquote{Internet API's}, and any other type of \iterm{external computing resource}.
Same restrictions for on-site external computing resources apply.

\noindent \textbf{Remark:} The competition organization doesn't guarantee or take any responsibility regarding the availability or reliability of neither the network nor Internet connection.
Teams' use of external computing resources is at their own risk.



% DSPL laptop
\subsection{Official Standard Laptop for DSPL}
\label{rule:osl_dspl}

In the Domestic Standard Platform League, teams must use the \iaterm{Official Standard Laptop}{OSL} connected to the Toyota HSR via Ethernet cable, safely located in the TOYOTA HSR \iterm{Mounting Bracket} provided by TOYOTA for this purpose.

Any laptop fitting inside the TOYOTA HSR \iterm{Mounting Bracket} is allowed, regardless of its technical specification.
All competing robots must have mounted an OSL, whether they use it or not, so all TOYOTA HSRs have the same load restrictions.

\subsubsection{External Wi-Fi adapter}
As described in section 6.2.2.5.6 of the Toyota HSR manual, an external Wi-Fi adapter can be used to increase stability and/or bandwidth of the network connection. As teams are allowed to use the network adapter of the \iterm{OSL}, teams may also chose to use an external Wi-Fi adapter.

No constraints are applied regards the Wi-Fi specifications. However to be used on the Toyota HSR, the Wi-Fi adapter should be USB-powered. To allow wired network connections between the internal computer, the \iterm{OSL} and the external Wi-Fi adapter, a (USB-powered) switch is allowed.

\noindent \textbf{Remark:} The usage of the network adapter of the \iterm{OSL} and/or the external Wi-Fi adapter doesn't allow for more than \emph{one} Wi-Fi connection per robot to the \iterm{arena network}.

% Local Variables:
% TeX-master: "../Rulebook"
% End:


%%%%%%%%%%%%%%%%%%%%%%%%%%%%%%%%%%%%%%%%%%%%%%%%%%%%%%%%%
\section{Organization of the competition}
\label{sec:procedure_during_competition}

\subsection{Stage system}\label{rule:stages}

The competition features a \iterm{stage system}. It is organized in two stages each consisting of a number of specific tasks. It ends with the \iterm{Finals}.

Each \iaterm{stage} comprehends a set of tasks grouped in two thematic scenarios.
% \iaterm{House Cleaner} and \iaterm{Party Host}.
The \iaterm{Housekeeper} scenario features tasks related to cleaning, organizing, and giving maintenance; while the \iaterm{Party Host} scenario focuses in attending guests needs and providing general assistance during a party.

\begin{enumerate}
	\item \textbf{Robot Inspection:} For security, robots are inspected during setup days.
  A robot must pass \iterm{Robot Inspection} test (see~\refsec{sec:robot_inspection}) in order to compete.

	\item \textbf{Stage~I:} The first days of the competition called \iterm{Stage~I}.
	All qualified teams can participate in \iterm{Stage~I}.
	The same task can be performed multiple times (See~\refsec{rule:score_system}).

	\item \textbf{Stage~II:} The best \emph{50\% of teams}\footnotemark (after Stage~I) advance to \iterm{Stage~II}.
	Here, tasks require more complex abilities or combinations of abilities.\\
	\footnotetext{If the total number of teams is less than 12, up to 6 teams may advance to Stage~II}

	\item \textbf{Final demonstration:} The best \emph{two teams} of each league, the ones with the highest score after Stage~II, advance to the final round.
	The final round features only a single task integrating all tested abilities.
	In order to participate in the Finals, a team must have solved at least one task of the Stage~II.
\end{enumerate}

In case of having no considerable score deviation between a team advancing to the next stage and a team dropping out, the TC may announce additional teams advancing to the next stage.


%%%%%%%%%%%%%%%%%%%%%%%%%%%%%%%%%%%%%%%%%%%%%%%%%%%%%%%%%
\subsection{Schedule}
\label{rule:schedule}

\begin{enumerate}
	\item \textbf{Thematic scenario blocks:} Each \iterm{thematic scenario} or \iterm{theme} is split in two \iterm{blocks}.
	At least two blocks are scheduled per day, having each block an assigned theme and lasting no less than two hours.
	The \iaterm{Organizing Committee}{OC} announces the schedule during the setup days (see Table \ref{tbl:schedule}).

	\item \textbf{Slots:} The \iaterm{Organizing Committee}{OC} assigns at least two \iterm{test slots} of 5 minutes to each team in each block.
   The maximum number of \iterm{tests slots} will be announced during setup days by the \iaterm{Technical Committee}{TC} based on the available time and the number of participating teams.
	A team can solve any task during its test slot.
	Remaining block time can be used to assign additional testing slots to interested teams.
	Testing slots are randomly assigned to teams in each block.

	\item \textbf{Tests:} Teams must inform the OC in advance which task(s) will try to solve.
	Only one task can be attempted per test slot.

	\item \textbf{Participation is default:} Teams have to indicate to the \iaterm{Organizing Committee}{OC} when they are \emph{skipping} a test slot. Without such indication, they may receive a penalty when not attending (see~\refsec{rule:not_attending}).
\end{enumerate}

% Please add the following required packages to your document preamble:
% \usepackage[table,xcdraw]{xcolor}
% If you use beamer only pass "xcolor=table" option, i.e. \documentclass[xcolor=table]{beamer}
\begin{table}[h]
	\centering\small
	\newcommand{\teams}[3]{%
		\tiny
		\begin{tabular}{c}%
			\textit{Test slot 1, team $#1$}\\
			\textit{Test slot 2, team $#2$}\\
			$\vdots$\\
			\textit{Test slot $n$, team $#3$}\\
		\end{tabular}
	}
	\newcommand{\wcell}[2]{%
		\parbox[c]{2.5cm}{%
			\vspace{#1}%
			\centering%
			#2%
			\vspace{#1}%
		}%
	}
	\newcommand{\cell}[1]{\wcell{0.2\baselineskip}{#1}}


	\begin{tabular}{
		>{\centering\arraybackslash}m{2.5cm}|%
		>{\columncolor[HTML]{9AFF99}}c |%
		>{\columncolor[HTML]{9AFF99}}c |%
		>{\columncolor[HTML]{CBCEFB}}c |%
		>{\columncolor[HTML]{CBCEFB}}c |%
	}
	\multicolumn{1}{ c }{}
		& \multicolumn{1}{ c }{\cellcolor{white} Day 1 }
		& \multicolumn{1}{ c }{\cellcolor{white} Day 2 }
		& \multicolumn{1}{ c }{\cellcolor{white} Day 3 }
		& \multicolumn{1}{ c }{\cellcolor{white} Day 4 }
		\\\cline{2-5}
	\cell{Block 1\\\footnotesize(9:00 - 12:00)}
		& \cell{Housekeeper\\\teams{i}{j}{i}}
		& \cell{Party Host\\\teams{k}{i}{k}}
		& \cell{Housekeeper\\\teams{i}{j}{i}}
		& \cell{Party Host\\\teams{j}{k}{j}}\\\cline{2-5}

	\multicolumn{1}{ c }{}
		& \multicolumn{4}{ c }{\wcell{0.5\baselineskip}{\color{gray}Lunch}}\\\cline{2-5}

	\cell{Block 2\\\footnotesize(14:00 - 17:00)}
		& \cell{Housekeeper\\\teams{i}{k}{i}}
		& \cell{Party Host\\\teams{k}{j}{k}}
		& \cell{Party Host\\\teams{i}{i}{k}}
		& \cell{Housekeeper\\\teams{k}{j}{j}}\\\cline{2-5}

	\multicolumn{1}{ c }{}
		& \multicolumn{2}{ c }{\wcell{0.5\baselineskip}{\color[HTML]{029734}Stage 1}}
		& \multicolumn{2}{ c }{\wcell{0.5\baselineskip}{\color[HTML]{6668e5}Stage 2}}\\
	\end{tabular}

	\caption{Example schedule.
		Each team has assigned at least two test slots in every block.
		At least two blocks are scheduled per day with an assigned theme.
		A team can choose a different task in each test, meaning at least 4 different tests per stage.
	}
	\label{tbl:schedule}
\end{table}


\subsection{Score system}
\label{rule:score_system}
Each task has a main objective and a set of scoring bonuses.
To score in a test, a team must successfully accomplish the main objective of the task; bonuses are not considered otherwise.

The overall score is calculated as the sum of the maximum scores obtained in each thematic scenario of each stage. For example, if a team scores 500 in \textit{Clean Up [Housekeeper]}, 400 in \textit{Take Out the Garbage [Housekeeper]} and 800 in \textit{Farewell [Party Host]}, their total score is $max([500, 400]) + max([800]) = 1300$ for Stage I.

The \iaterm{score system} has the following constrains
\begin{enumerate}

	\item \textbf{Stage~I:} The maximum total score per task in \iterm{Stage~I} is \scoring{1000 points}.
	
	\item \textbf{Stage~II:} The maximum total score per task in \iterm{Stage~I} is \scoring{2000 points}.

	\item \textbf{\iterm{Finals}:} Final score is normalized and a special evaluation is used.

	\item \textbf{Minimum score:} The minimum total score per test in \iterm{Stage~I} and \iterm{Stage~II} is \scoring{0 points}.
	Teams cannot receive negative points.

	\item \textbf{Penalties:} An exception to \emph{minimum score} rule are penalties.
	Both penalties for not attending (see~\refsec{rule:not_attending}) and extraordinary penalties (see~\refsec{rule:extraordinary_penalties}) can cause a total negative score.
\end{enumerate}




% Local Variables:
% TeX-master: "../Rulebook"
% End:


%%%%%%%%%%%%%%%%%%%%%%%%%%%%%%%%%%%%%%%%%%%%%%%%%%%%%%%%%
\section{Procedure during Tests}

\subsection{Safety First!}
\label{rule:safetyfirst}
\begin{enumerate}
	\item \textbf{Emergency Stop:} At any time when operating the robot inside and outside the scenario the owners have to stop the robot immediately if there is a remote possibility of dangerous behavior towards people and/or objects. 
	\item \textbf{Stopping on request:} If a referee, member of the Technical or Organizational committee, an Executive or Trustee of the federation tells the team to stop the robot, there will be no discussion and the robot has to be stopped \emph{immediately}.
	\item \textbf{Penalties:} If the team does not comply, the team and its members will be excluded from the ongoing competition immediately by a decision of the RoboCup@Home \iaterm{Technical Committee}{TC}. 	Furthermore, the team and its members may be banned from future competitions for a period not less than a year by a decision of the RoboCup Federation Trustee Board.
\end{enumerate}

\subsection{Maximum number of team members}
\label{rule:number_of_people}
\begin{enumerate}
	\item \textbf{Regular Tests:} During a regular test, the maximum number of team members allowed inside the arena is \emph{one} (1). The only exceptions are tests that require for more team members in the arena.
	\item \textbf{Setup:} During the setup of a test, the number of team members inside the arena is not limited.
	% \item \textbf{Open Demonstrations:} During the \iterm{Open Challenge} \iterm{Demo Challenge}, and the \iaterm{final demonstration}{Finals}, the number of team members inside the arena is not limited.
	\item \textbf{Open Demonstrations:} During the \iterm{Open Challenge}, and the \iaterm{final demonstration}{Finals}, the number of team members inside the arena is not limited. 
	\item \textbf{Moderation:} During a regular test, one team member \emph{must} be available to host and comment the event (see \refsec{rule:moderator}).
\end{enumerate}

\subsection{Fair play}
\label{rule:fairplay}
\iterm{Fair Play} and cooperative behavior is expected from all teams during the entire competition, in particular:
\begin{itemize}
	\item while evaluating other teams, 
	\item while refereeing, and 
	\item when having to interact with other teams' robots.  
\end{itemize}
This also includes:
\begin{itemize}
	\item not trying to cheat (e.g.~pretending autonomous behavior where there is none), 
	\item not trying to exploit the rules (e.g.~not trying to solve the task but trying to score), and 
	\item not trying to make other robots fail on purpose. 
	\item not modifying robots in standard platforms. 
\end{itemize}
Disregard of this rule can lead to penalties in the form of negative scores, and disqualification for a test or even for the entire competition. 

\subsection{Expected Robot's Behavior}
Unless stated otherwise, it is expected that the robot always behave and react in the same way a polite and friendly human being would do. This applies also to how robots should address the problems in order to solve the assigned task, including addressing people, serving meals, storing the groceries, cleaning, arranging stuff, etc. As rule of thumb, ask your closest non-scientist neighbor to solve the task and take notes.

Keep in mind that one of the goals in RoboCup @Home is to have robots interacting with real people in domestic environments. This means that the average user will not know any specific procedure on how to operate the robot, but they will interact with it as with any other human being.


\subsection{Robot Autonomy and Remote Control}
\begin{enumerate}
	\item \textbf{No touching:} During a test, the participants are not allowed to make contact with the robot(s), unless it is in a \quotes{natural} way and/or required by the test specification. 
	\item \textbf{Natural interaction:} The only allowed means to interact with the robot(s) are gestures and speech.
	\item \textbf{Natural commands:} Only general instructions are allowed. 
	Anything that resembles direct control is prohibited.
	\item \textbf{Remote Control:} Remotely controlling the robot(s) is strictly prohibited. This also includes pressing buttons, or influencing sensors on purpose.
	\item \textbf{Penalties:} Disregard of these rules can lead to penalties in the form of negative scores, and disqualification for a test or even for the entire competition. 
\end{enumerate}

\subsection{Collisions}
\begin{enumerate}
	\item \textbf{\iterm{Touching}:} Robots are allowed to gently \emph{touch} objects, items and humans. 	They are not allowed to crash into something. The \quotes{safety first} rule (\refsec{rule:safetyfirst}) supercedes all other rules.
	\begin{itemize}
		\item It \emph{is} allowed however to \emph{functionally} touch an item with e.g.~the base.
	\end{itemize}
	The OC/TC/EC and the RoboCup Trustees all have the right to immediately stop a robot, and to disqualify a team for the duration of the competition, or longer, in case of \emph{dangerous} behavior. Furthermore, referees can recommend to disqualify a team in which case EC/TC decides.
	\item \textbf{\iterm{Major collisions}:} If a robot crushes into something during a test, the robot is immediately stopped.	Additional penalties may apply. 
	\item \textbf{Robot-Robot avoidance:} If two robots encounter each other, they both have to actively try to avoid the other robot.
	\begin{enumerate}
		\item A robot which is not going for a different route within a reasonable amount of time (e.g., \SI{30}{\second}) is removed.
		\item A non-moving robot blocking the path of another robot for longer than a reasonable amount of time (e.g., \SI{30}{\second}) is removed. In this context, \quotes{moving} refers to any kind of motion or action required in the test. For example, a robot standing still but manipulating an object does not need to stop manipulating and move away, even when blocking the way of another robot for the duration of the manipulation.
	\end{enumerate}
\end{enumerate}



\subsection{Removal of robots}
\label{rule:robot_removal}
Robots not obeying the rules are stopped and removed from the arena.
\begin{enumerate}
	\item It is the decision of the referees and the TC member monitoring the test if and when to remove a robot.
	\item When told to do so by the referees or the TC member monitoring the test, the team has to immediately stop the robot, and remove it from the arena without disturbing the ongoing test.
\end{enumerate}


\subsection{Start signal}
\label{rule:start_signal}

Different tests are started in different ways, according to what would make the most sense in the application setting. 
Before a test starts, robots are waiting in a queue, sometimes accompanied by a team member. 

The various start methods are described below:
\begin{enumerate}
	\item \textbf{Door opening:} The robot is waiting behind the door, outside the arena (accompanied by a team member). The test attempt starts when a referee (not a team member) opens the door. 
	\item \textbf{Start button:} If the robot is not able to automatically start after opening the door, the team may start the robot using a start button. 
	\begin{enumerate}
		\item Using a start button needs to be announced to the referees. It is the responsibility of the team to do so before the test starts.
		\item There may be penalties for using a start button in some tests
	\end{enumerate}
    \item \textbf{Called by name:} A number of robots is waiting inside the arena, unaccompanied by team members. 
				      The referee approaches the robot, calls it by its name and gives the robot a command. e.g. \quotes{R2D2, start} or \quotes{C3PO, continue}.
				      Other waiting robots must not respond. 
\end{enumerate}


\subsection{Entering and leaving the arena}
\label{rule:start_position}
\begin{enumerate}
	\item \textbf{Start position:} Unless stated otherwise, the robot starts outside of the arena.
	\item \textbf{Entering:} The robot has to autonomously enter the arena.
	\item \textbf{Success:} The robot is said to \emph{have entered} when the door used to enter can be closed again, and the robot is not blocking the passage.
\end{enumerate}



\subsection{Gestures}
\label{rule:gestures}
Hand gestures may be used to control the robot in the following way:
\begin{enumerate}
	\item \textbf{Definition:} The teams define the hand gestures by themselves. 
	\item \textbf{Approval:} Gestures need to be approved by the referees and TC member monitoring the test. Gestures should not involve more than the movement of both arms. This includes e.g.~expressions of sign language or pointing gestures.
	\item \textbf{Instructing operators:} It is the responsibility of the team to instruct operators.
	\begin{enumerate}
		\item The team may only instruct the operator when told to so by a referee.
		\item The team may only instruct the operator in the presence of a referee.
		\item The team may only instruct the robot for as long as allowed by the referee.
		\item When the robot has to instruct the operator, it is the robot that instructs the operator and \emph{not} the team. The team is not allowed to additionally guide the operator, e.g., tell the operator to come closer, speak louder, or to repeat a command.
		\item The robot is allows to instruct the operator at any time.
	\end{enumerate}
	\item \textbf{Receiving gestures:} Unless stated otherwise, it is not allowed to use a speech command to set the robot into a special mode for receiving gestures.
\end{enumerate}



\subsection{Referees}
\label{rule:referees}
% \refmark{}
\begin{enumerate}
	\item \textbf{Setup:} Unless stated otherwise, each test is monitored by two referees and one member of the \iaterm{Technical Committee}{TC}.
	\item \textbf{Selection:} The two referees 
	\begin{itemize}
		\item are chosen by EC/TC/OC, 
		\item are announced together with the schedule for the test slot, 
		\item and have to referee all teams in that slot.
		\item Referees may not be from one of the teams in the slot.
	\end{itemize}
	\item \textbf{Not showing up:} Not showing up for refereeing (on time) will result in a penalty (see \refsec{rule:extraordinary_penalties}). 
	\item \textbf{TC monitoring:} The referee from the TC acts as a main referee. 
	\item \textbf{Referee instructions:} Right before each test, referee instructions are conducted by the TC. The referees for all slots need to be present at the arena where the referee instructions are taking place.When and where referee instructions are taking place is announced together with the schedule for the slots.
\end{enumerate}


\subsection{Operator}
\label{rule:operator}
\begin{enumerate}
	\item \textbf{Default operator:} Unless stated otherwise, robots are operated by the monitoring TC member, a referee, or by a person selected by the TC.
	\item \textbf{Fallback/custom operator:} If the robot fails to understand the command given by the default operator, the team may continue with a custom operator.
	\begin{compactitem}
		\item The custom operator may be any person chosen by the team (and willing to do so); including the referees or the monitoring TC member. 
		\item A penalty may be involved when using a custom operator.
	\end{compactitem}
\end{enumerate}



\subsection{Moderator}
\label{rule:moderator}
\begin{enumerate}
	\item \textbf{Providing a moderator:} For each regular test (i.e., not for the open demonstrations), all participating teams need to provide a team member as moderator for the duration of their performance. 
	\item \textbf{Responsibilities:} The moderators have to:
	\begin{compactitem}
		\item explain the rules of the test, 
		\item comment on the performance of their team, 
		\item not interfere with the performance, 
		\item speak in English, 
		\item and obey the instructions by the monitoring TC member.
	\end{compactitem}
	\item \textbf{Competitive tests:} In competitive tests (tests in which two teams directly compete against each other), the moderation has to be done by the two teams together.
\end{enumerate}


\subsection{Time limits}
\label{rule:time_limits}
\begin{enumerate}
	\item \textbf{Stage~I:} Unless stated otherwise, the time limit for each test in \iterm{Stage~I} is \timing{5 minutes}.
	\item \textbf{Stage~II:} Unless stated otherwise, the time limit for each test in \iterm{Stage~II} is \timing{10 minutes}.
	\item \textbf{Setup time:} Unless stated otherwise, all time specifications, e.g., setup time and time for instructing operators, are within the total test time. 
	\item \textbf{Scores:} When the time is up, the team has to immediately remove their robot(s) from the arena; no more points can be scored. In special cases, the monitoring TC member may ask the team to continue the test for demonstration purposes (after time is up, points cannot be scored). 
\end{enumerate}



\subsection{Restart}
\label{rule:restart}
\begin{enumerate}
	\item \textbf{Stage 1} has no restarts but features multiple attempts at a test. 
	If a robot fails during an attempt, the attempt ends. An attempt cannot be restarted. 
	E.g. if a robot fails halfway through an attempt at the Help-me-Carry test, the attempt is over, the robot is moved out of the test area and may prepare for the remaining attempts at the test.

	\item \textbf{Stage 2} does have restarts for all tests but the Open Challenge:
	\begin{enumerate}
		\item \textbf{Number of restarts:} A team may request one (1) restart during a test, unless stated otherwise. There are tests in which a restart is not allowed.
		\item \textbf{Procedure:} In case a restart is allowed, the team may request the restart only before 50\% of the time alloted to the test. The complete test is then restarted from the beginning (e.g., with entering the arena). The referees may rearrange the locations of objects/persons if necessary.
		\item \textbf{Time:} The time is neither restarted nor stopped. The team has 1 minute to restart the test (the same time to start the test); if the team is not able to do so in the allotted time, the test is called as finished by the TC.
		\item \textbf{Score:} The score of the second run (after the restart) counts. If it is lower than the score of the first run (before the restart), the average score of first and second run is taken.
		\item \textbf{Forced restart:} The referees and the monitoring TC member may force the team to do a restart:
		\begin{compactitem}
			\item if the robot is doing nothing or nothing reasonable for \timing{one minute}, or
			\item when the robot fails to understand a command for \timing{five times}, or
			\item after a minor collision
		\end{compactitem}  
	\end{enumerate}
\end{enumerate}

% Local Variables:
% TeX-master: "../Rulebook"
% End:

\input{general_rules/ContinueRules.tex}

%%%%%%%%%%%%%%%%%%%%%%%%%%%%%%%%%%%%%%%%%%%%%%%%%%%%%%%%%
\newcommand{\penaltybig}{50~}
\newcommand{\penaltysmall}{50~}


\section{Special penalties and bonuses}\label{sec:special_awards}

\subsection{Penalty for not attending}\label{rule:not_attending}
\begin{enumerate}
	\item \textbf{Automatic schedule:} All teams are automatically scheduled for all tests.
	\item \textbf{Announcement:} If a team cannot participate in a test (for any reason), the team leader has to announce this to the OC at least \timing{60 minutes} before the test slot begins.
	\item \textbf{Penalties:} A team that is not present at the start position when their scheduled test starts, the team is not allowed to participate in the test anymore. If the team has not announced that it is not going to participate, it gets a penalty of \scoring{\penaltybig points}.
\end{enumerate}

\subsection{Extraordinary penalties}\label{rule:extraordinary_penalties}
\begin{enumerate}
	\item \textbf{Penalty for faking robots:} If a team starts a test, but it does not solve any of the partial tasks (and is obviously not trying to do so), a penalty of \scoring{\penaltybig points} is handed out. The decision is made by the referees and the monitoring TC member.

	\item \textbf{Extra penalty for collision:} In case of major, (grossly) negligent collisions the \iaterm{Technical Committee}{TC} may disqualify the team for a test (the team receives \scoring{0 points}), or for the entire competition.

	\item \textbf{Not showing up as referee or jury member:} If a team does not provide a referee or jury member (being at the arena on time), the team receives a penalty of \scoring{\penaltybig points}, and will be remembered for qualification decisions in future competitions.\\
	Jury members missing a performance to evaluate are excluded from the jury, and the team is disqualified from the test (receives \scoring{0 points}).

	\item \textbf{Modifying or altering standard platform robots:} If any unauthorized modification is found on a Standard Platform League robot, the responsible team will be immediately disqualified for the entire competition while also receiving a penalty of \scoring{\penaltybig points} in the overall score. This behavior will be remembered for qualification decisions in future competitions.\\
\end{enumerate}

\subsection{Bonus for outstanding performance}\label{rule:outstanding_performance}
\begin{enumerate}
	\item For every regular test in \iterm{Stage~I} and \iterm{Stage~II}, the @Home \iaterm{Technical Committee}{TC} can decide to give an extra bonus for \iterm{outstanding performance} of up to 10\% of the maximum test score.
	\item This is to reward teams that do more than what is needed to solely score points in a test but show innovative and general approaches to enhance the scope of @Home.
	\item If a team thinks that it deserves this bonus, it should announce (and briefly explain) this to the \iaterm{Technical Committee}{TC} beforehand.
	\item It is the decision of the \iaterm{Technical Committee}{TC} if (and to which degree) the bonus score is granted.
\end{enumerate}


% Local Variables:
% TeX-master: "../Rulebook"
% End:



% Local Variables:
% TeX-master: "Rulebook"
% End:


\input{Setup}

\chapter{Tests in Stage I}
\label{chap:stage_I}

\begin{itshape}
\iterm{Stage~I} comprehends three \textbf{ability tests} and an \textbf{integration test}.
Each ability test is designed to evaluate the average performance of the robot in one particular skill
% , providing data for benchmarking.
Meanwhile, the integration test has been designed to evaluate how this abilities work together while solving a common task.

The total score for ability and integration tests is the average of the best two performances out of preferably three performances (given the time constraints of a competition).
The point of this is to both eliminate good and bad luck for the robots/teams and to get a more objective view of the performance,
  not to give teams time to tweak the robot between test performances.

\iterm{Help-me-carry} (demonstration for the audience) goes out of the arena and into the venue between the audience.

\end{itshape}

\subsection*{Scheduling}
For maximal efficiency, teams will be scheduled interleaved:
  Team A does an attempt while team B sets up their robot. When A is done, it moves out the way for team B, then B attempts while A sets up the robot again etc.

The preparing team should prepare their robot close to the place of the test, but not interfere with the performing robot.
Prepared robots must wait at this preparation location until commanded to start the test.
When commanded to start, the robot must move automatically beyond this point.

Robot should be ready to start the next attempt to the same test as fast as possible:
  when the performing robot is done with a attempt, the next robot must be ready to go with the start of a button or a voice command.

\newpage
\section{Cocktail Party [SSPL only]}

The robot has to learn and recognize previously unknown people, and fetch orders.

\subsection{Focus}

This test focuses on human detection and recognition, safe navigation and human-robot interaction with unknown people.

\subsection{Setup}
\begin{itemize}
	\item \textbf{Party room}: any (large) room inside the apartment when normally a party would be held.
	\item \textbf{Guests:} At least five people are distributed in a predefined \quotes{party room} either sitting or standing, some of them forming groups of 2 or 3 people, from which at least one is sitting. Three of the guests have drink orders assigned by the referees. The sitting person always has an assigned drink.
	\item \textbf{Bar:} The bar is any flat surface where objects can be placed, in a room other than the \quotes{party room}. All available beverages are on top of the bar for the robot can see them.
	\item \textbf{Barman:} The Barman bay be standing either behind the bar or next to it, depending on the arena setup.
\end{itemize}

\subsection{Task}

\begin{enumerate}

	\item \textbf{Entering:} The robot enters the arena and navigates to the party room and waits for being called.

	\item \textbf{Getting called:} The guests call the robot simultaneously, either rising an arm, waving, or shouting. The robot has to approach one of them.
	% Remark: Themself is the correct word used instead of himself/herself  when person gender is unknown.
	The calling person introduces themself by name before giving the order of a drink. The robot leads the dialogue to learn the person and retrieve their drink order. \\

	The robot can decide to skip the detection of the calling and ask one person to walk in front of it. In this case, the referees determine the person to approach the robot.

	\item \textbf{Taking the order:} After the robot has fetched the order of the first guest, it can either fetch more orders (i.e.~ find next calling guest or looking for the sitting one) or proceed to place the order. In the first case, the robot searches for the remaining calling people. During the search process, the robot is allowed to either ask people to call for it again, or to ask people to come to it and to give a new order. In both cases the robot may call into the room.

	\item \textbf{Sitting person:} At least one person is sitting and minding their own business (i.e. not looking directly to the robot). The robot must locate that person and ask if the person wants something to drink. The robot must also ask for the person's name and memorize them (i.e. execute a learning procedure of the name and the person's features). \\

	\textbf{Remark:} At least one sitting person has an order to place. No sitting person will attend to the robot's call
	(i.e. robot has to approach to them).

	\item \textbf{Placing the orders:} The robot has to navigate to the \textit{Bar}, a designated location in another room where drinks are served. The robot must repeat each order to the \textit{Barman}, clearly stating:
	\begin{enumerate}
		\item The person's name,
		\item The person's chosen drink,
		\item A description of unique characteristics of that person that allow the \textit{Barman} to find them (e.g. gender, hair colour, how is dressed, etc).
	\end{enumerate}

	While the robot places the orders, the people in the \quotes{party room} may change their places within the party room (on request of the referees).

	\item \textbf{Missing beverage:} One of the ordered drinks is not available, therefore, missing from the bar. The robot should realize this inconvenience and tell the \textit{Barman}, providing a list of 3 alternatives considering the other drinks it needs to deliver. If the robot can't detect which drink is missing, the \textit{Barman} will clearly state which of the beverages is not available and provide a list of 3 alternatives.

	\item \textbf{Correcting an order:} The robot should navigate back to the \quotes{party room}, find the person whose drink is missing and provide the alternatives for they to choose.\\

	If the robot comes to the place the person ordered and the person is not there, it can call that person loud, the person should respond (either sound or waving hand) and the robot must go to that place (check the person identity).

	\item \textbf{Placing the corrected order:} The robot has to navigate to the \textit{Bar} and inform the change on the guest's order. This time only the guest's name and drink can be provided.
\end{enumerate}

\subsection{Additional rules and remarks}
\begin{enumerate}
	\item \textbf{Repeating names:} The robot may ask to repeat the name if it has not understood it.

	\item \textbf{Misunderstood names:} If the robot misunderstands the name, the understood (wrong) name is used in the remainder of this test.

	\item \textbf{Misunderstood order:} If the robot does not understand the order, it can continue with an own assignment of drinks to people or with a wrong, misunderstood assignment.

	\item \textbf{Approaching non-calling people:} If the robot approaches a person that is not calling and asks for an order, the person indicates that they does not want to order anything. No points can be scored for understanding names or orders, or for grasping or delivery for a non-calling person.

	\item \textbf{Guest description:} The guest's description must be unique inside the scenario. For instance, it make no sense to state that a person is wearing a red T-shirt if two people are wearing them. In the same sense, stating that the ordering guest is \textit{tall} can lead to confusion, but stating that is the \textit{tallest} does not.

	\item \textbf{Changing places:} After giving the order (when the robot is not in the party room), the referees may re-arrange the people including their body posture. That is, a sitting person may change to a standing posture and vice versa.

	\item \textbf{Positions and orientations:} All people roughly stay where they are (if not asked to move by the referees), but they are allowed to move in certain limits (e.g. turn around, make a step aside). They do not need to look at the robot, but are requested to do so, when instructed by the robot.

	\item \textbf{Empty arena:} During the test, only the robot, the guest, and the Barman are in the arena. The door opener, the referees and other personnel that is not assigned as test people will be outside the scenario.

	\item \textbf{Calling instruction:} The team needs to specify before the test which ways of getting the attention of the robot are allowed. This can be waving, calling or both of them. The robot can also decide to skip this part, by asking for people to get close to it.
\end{enumerate}

\subsection{Referee instructions}

The referees need to
\begin{itemize}
	\item select 5 people and their names from the list of person names (see Section 3.2.8),
	\item arrange (and re-arrange) people in the party room,
	\item select the person (barman) who will serve the drinks,
	\item select the ordering 3 people and the orders to give,
	\item in case the robot skips the calling detection, select the ordering person to approach the robot,
	\item write down the understood names and drinks during an order and update the order accordingly.
\end{itemize}

\subsection{OC instructions}

2h before test:
\begin{itemize}
	\item Specify and announce the rooms where the test takes place.
	\item Specify and announce the location where the drinks are served (i.e. bar location).
\end{itemize}

\newpage
\subsection{Score sheet}
\section{Cocktail Party [SSPL only]}

The robot has to learn and recognize previously unknown people, and fetch orders.

\subsection{Focus}

This test focuses on human detection and recognition, safe navigation and human-robot interaction with unknown people.

\subsection{Setup}
\begin{itemize}
	\item \textbf{Party room}: any (large) room inside the apartment when normally a party would be held.
	\item \textbf{Guests:} At least five people are distributed in a predefined \quotes{party room} either sitting or standing, some of them forming groups of 2 or 3 people, from which at least one is sitting. Three of the guests have drink orders assigned by the referees. The sitting person always has an assigned drink.
	\item \textbf{Bar:} The bar is any flat surface where objects can be placed, in a room other than the \quotes{party room}. All available beverages are on top of the bar for the robot can see them.
	\item \textbf{Barman:} The Barman bay be standing either behind the bar or next to it, depending on the arena setup.
\end{itemize}

\subsection{Task}

\begin{enumerate}

	\item \textbf{Entering:} The robot enters the arena and navigates to the party room and waits for being called.

	\item \textbf{Getting called:} The guests call the robot simultaneously, either rising an arm, waving, or shouting. The robot has to approach one of them.
	% Remark: Themself is the correct word used instead of himself/herself  when person gender is unknown.
	The calling person introduces themself by name before giving the order of a drink. The robot leads the dialogue to learn the person and retrieve their drink order. \\

	The robot can decide to skip the detection of the calling and ask one person to walk in front of it. In this case, the referees determine the person to approach the robot.

	\item \textbf{Taking the order:} After the robot has fetched the order of the first guest, it can either fetch more orders (i.e.~ find next calling guest or looking for the sitting one) or proceed to place the order. In the first case, the robot searches for the remaining calling people. During the search process, the robot is allowed to either ask people to call for it again, or to ask people to come to it and to give a new order. In both cases the robot may call into the room.

	\item \textbf{Sitting person:} At least one person is sitting and minding their own business (i.e. not looking directly to the robot). The robot must locate that person and ask if the person wants something to drink. The robot must also ask for the person's name and memorize them (i.e. execute a learning procedure of the name and the person's features). \\

	\textbf{Remark:} At least one sitting person has an order to place. No sitting person will attend to the robot's call
	(i.e. robot has to approach to them).

	\item \textbf{Placing the orders:} The robot has to navigate to the \textit{Bar}, a designated location in another room where drinks are served. The robot must repeat each order to the \textit{Barman}, clearly stating:
	\begin{enumerate}
		\item The person's name,
		\item The person's chosen drink,
		\item A description of unique characteristics of that person that allow the \textit{Barman} to find them (e.g. gender, hair colour, how is dressed, etc).
	\end{enumerate}

	While the robot places the orders, the people in the \quotes{party room} may change their places within the party room (on request of the referees).

	\item \textbf{Missing beverage:} One of the ordered drinks is not available, therefore, missing from the bar. The robot should realize this inconvenience and tell the \textit{Barman}, providing a list of 3 alternatives considering the other drinks it needs to deliver. If the robot can't detect which drink is missing, the \textit{Barman} will clearly state which of the beverages is not available and provide a list of 3 alternatives.

	\item \textbf{Correcting an order:} The robot should navigate back to the \quotes{party room}, find the person whose drink is missing and provide the alternatives for they to choose.\\

	If the robot comes to the place the person ordered and the person is not there, it can call that person loud, the person should respond (either sound or waving hand) and the robot must go to that place (check the person identity).

	\item \textbf{Placing the corrected order:} The robot has to navigate to the \textit{Bar} and inform the change on the guest's order. This time only the guest's name and drink can be provided.
\end{enumerate}

\subsection{Additional rules and remarks}
\begin{enumerate}
	\item \textbf{Repeating names:} The robot may ask to repeat the name if it has not understood it.

	\item \textbf{Misunderstood names:} If the robot misunderstands the name, the understood (wrong) name is used in the remainder of this test.

	\item \textbf{Misunderstood order:} If the robot does not understand the order, it can continue with an own assignment of drinks to people or with a wrong, misunderstood assignment.

	\item \textbf{Approaching non-calling people:} If the robot approaches a person that is not calling and asks for an order, the person indicates that they does not want to order anything. No points can be scored for understanding names or orders, or for grasping or delivery for a non-calling person.

	\item \textbf{Guest description:} The guest's description must be unique inside the scenario. For instance, it make no sense to state that a person is wearing a red T-shirt if two people are wearing them. In the same sense, stating that the ordering guest is \textit{tall} can lead to confusion, but stating that is the \textit{tallest} does not.

	\item \textbf{Changing places:} After giving the order (when the robot is not in the party room), the referees may re-arrange the people including their body posture. That is, a sitting person may change to a standing posture and vice versa.

	\item \textbf{Positions and orientations:} All people roughly stay where they are (if not asked to move by the referees), but they are allowed to move in certain limits (e.g. turn around, make a step aside). They do not need to look at the robot, but are requested to do so, when instructed by the robot.

	\item \textbf{Empty arena:} During the test, only the robot, the guest, and the Barman are in the arena. The door opener, the referees and other personnel that is not assigned as test people will be outside the scenario.

	\item \textbf{Calling instruction:} The team needs to specify before the test which ways of getting the attention of the robot are allowed. This can be waving, calling or both of them. The robot can also decide to skip this part, by asking for people to get close to it.
\end{enumerate}

\subsection{Referee instructions}

The referees need to
\begin{itemize}
	\item select 5 people and their names from the list of person names (see Section 3.2.8),
	\item arrange (and re-arrange) people in the party room,
	\item select the person (barman) who will serve the drinks,
	\item select the ordering 3 people and the orders to give,
	\item in case the robot skips the calling detection, select the ordering person to approach the robot,
	\item write down the understood names and drinks during an order and update the order accordingly.
\end{itemize}

\subsection{OC instructions}

2h before test:
\begin{itemize}
	\item Specify and announce the rooms where the test takes place.
	\item Specify and announce the location where the drinks are served (i.e. bar location).
\end{itemize}

\newpage
\subsection{Score sheet}
\section{Cocktail Party [SSPL only]}

The robot has to learn and recognize previously unknown people, and fetch orders.

\subsection{Focus}

This test focuses on human detection and recognition, safe navigation and human-robot interaction with unknown people.

\subsection{Setup}
\begin{itemize}
	\item \textbf{Party room}: any (large) room inside the apartment when normally a party would be held.
	\item \textbf{Guests:} At least five people are distributed in a predefined \quotes{party room} either sitting or standing, some of them forming groups of 2 or 3 people, from which at least one is sitting. Three of the guests have drink orders assigned by the referees. The sitting person always has an assigned drink.
	\item \textbf{Bar:} The bar is any flat surface where objects can be placed, in a room other than the \quotes{party room}. All available beverages are on top of the bar for the robot can see them.
	\item \textbf{Barman:} The Barman bay be standing either behind the bar or next to it, depending on the arena setup.
\end{itemize}

\subsection{Task}

\begin{enumerate}

	\item \textbf{Entering:} The robot enters the arena and navigates to the party room and waits for being called.

	\item \textbf{Getting called:} The guests call the robot simultaneously, either rising an arm, waving, or shouting. The robot has to approach one of them.
	% Remark: Themself is the correct word used instead of himself/herself  when person gender is unknown.
	The calling person introduces themself by name before giving the order of a drink. The robot leads the dialogue to learn the person and retrieve their drink order. \\

	The robot can decide to skip the detection of the calling and ask one person to walk in front of it. In this case, the referees determine the person to approach the robot.

	\item \textbf{Taking the order:} After the robot has fetched the order of the first guest, it can either fetch more orders (i.e.~ find next calling guest or looking for the sitting one) or proceed to place the order. In the first case, the robot searches for the remaining calling people. During the search process, the robot is allowed to either ask people to call for it again, or to ask people to come to it and to give a new order. In both cases the robot may call into the room.

	\item \textbf{Sitting person:} At least one person is sitting and minding their own business (i.e. not looking directly to the robot). The robot must locate that person and ask if the person wants something to drink. The robot must also ask for the person's name and memorize them (i.e. execute a learning procedure of the name and the person's features). \\

	\textbf{Remark:} At least one sitting person has an order to place. No sitting person will attend to the robot's call
	(i.e. robot has to approach to them).

	\item \textbf{Placing the orders:} The robot has to navigate to the \textit{Bar}, a designated location in another room where drinks are served. The robot must repeat each order to the \textit{Barman}, clearly stating:
	\begin{enumerate}
		\item The person's name,
		\item The person's chosen drink,
		\item A description of unique characteristics of that person that allow the \textit{Barman} to find them (e.g. gender, hair colour, how is dressed, etc).
	\end{enumerate}

	While the robot places the orders, the people in the \quotes{party room} may change their places within the party room (on request of the referees).

	\item \textbf{Missing beverage:} One of the ordered drinks is not available, therefore, missing from the bar. The robot should realize this inconvenience and tell the \textit{Barman}, providing a list of 3 alternatives considering the other drinks it needs to deliver. If the robot can't detect which drink is missing, the \textit{Barman} will clearly state which of the beverages is not available and provide a list of 3 alternatives.

	\item \textbf{Correcting an order:} The robot should navigate back to the \quotes{party room}, find the person whose drink is missing and provide the alternatives for they to choose.\\

	If the robot comes to the place the person ordered and the person is not there, it can call that person loud, the person should respond (either sound or waving hand) and the robot must go to that place (check the person identity).

	\item \textbf{Placing the corrected order:} The robot has to navigate to the \textit{Bar} and inform the change on the guest's order. This time only the guest's name and drink can be provided.
\end{enumerate}

\subsection{Additional rules and remarks}
\begin{enumerate}
	\item \textbf{Repeating names:} The robot may ask to repeat the name if it has not understood it.

	\item \textbf{Misunderstood names:} If the robot misunderstands the name, the understood (wrong) name is used in the remainder of this test.

	\item \textbf{Misunderstood order:} If the robot does not understand the order, it can continue with an own assignment of drinks to people or with a wrong, misunderstood assignment.

	\item \textbf{Approaching non-calling people:} If the robot approaches a person that is not calling and asks for an order, the person indicates that they does not want to order anything. No points can be scored for understanding names or orders, or for grasping or delivery for a non-calling person.

	\item \textbf{Guest description:} The guest's description must be unique inside the scenario. For instance, it make no sense to state that a person is wearing a red T-shirt if two people are wearing them. In the same sense, stating that the ordering guest is \textit{tall} can lead to confusion, but stating that is the \textit{tallest} does not.

	\item \textbf{Changing places:} After giving the order (when the robot is not in the party room), the referees may re-arrange the people including their body posture. That is, a sitting person may change to a standing posture and vice versa.

	\item \textbf{Positions and orientations:} All people roughly stay where they are (if not asked to move by the referees), but they are allowed to move in certain limits (e.g. turn around, make a step aside). They do not need to look at the robot, but are requested to do so, when instructed by the robot.

	\item \textbf{Empty arena:} During the test, only the robot, the guest, and the Barman are in the arena. The door opener, the referees and other personnel that is not assigned as test people will be outside the scenario.

	\item \textbf{Calling instruction:} The team needs to specify before the test which ways of getting the attention of the robot are allowed. This can be waving, calling or both of them. The robot can also decide to skip this part, by asking for people to get close to it.
\end{enumerate}

\subsection{Referee instructions}

The referees need to
\begin{itemize}
	\item select 5 people and their names from the list of person names (see Section 3.2.8),
	\item arrange (and re-arrange) people in the party room,
	\item select the person (barman) who will serve the drinks,
	\item select the ordering 3 people and the orders to give,
	\item in case the robot skips the calling detection, select the ordering person to approach the robot,
	\item write down the understood names and drinks during an order and update the order accordingly.
\end{itemize}

\subsection{OC instructions}

2h before test:
\begin{itemize}
	\item Specify and announce the rooms where the test takes place.
	\item Specify and announce the location where the drinks are served (i.e. bar location).
\end{itemize}

\newpage
\subsection{Score sheet}
\input{scoresheets/CocktailParty.tex}




\newpage
\section{General Purpose Service Robot [Housekeeper]}
\label{test:gpsr}
Similar to a modern smart-speaker, the robot can be asked to do anything from the Stage~I of this rulebook or any previous rulebook.

% \subsection*{Focus}
% This test focuses on the detection and recognition of objects and their features, as well as object manipulation.

\subsection*{Main Goal}
Execute each of the 3 commands requested by the operator.

\noindent\textbf{Reward:} 750pts (250 points per command)\\

\subsection*{Bonus rewards}
\begin{enumerate}[nosep]
	\item Understand a command given by naive operator (50pts, each).
	\item Provide audio recording and transcript (100pts each).
	\item Autonomously leaving the arena (150pts).
\end{enumerate}

% %% %%%%%%%%%%%%%%%%%%%%%%%%%%%%%%%%%%%%%%%%%%%%%%%%%%%%%%
%
% Setup
%
% %% %%%%%%%%%%%%%%%%%%%%%%%%%%%%%%%%%%%%%%%%%%%%%%%%%%%%%%
\subsection*{Setup}
\begin{enumerate}[nosep]
	\item \textbf{Location:} The task takes place inside the arena (some commands might require the robot to go out). The arena is in its normal state.

	\item \textbf{Start location:} The robot starts outside the arena. When the door opens, the robot moves towards the \textit{Instruction Point}.

	\item \textbf{Operators:} A \emph{professional operator} (i.e.~the referee) commands the robot to execute a task.
\end{enumerate}


% %% %%%%%%%%%%%%%%%%%%%%%%%%%%%%%%%%%%%%%%%%%%%%%%%%%%%%%%
%
% Additional Rules
%
% %% %%%%%%%%%%%%%%%%%%%%%%%%%%%%%%%%%%%%%%%%%%%%%%%%%%%%%%
\subsection*{Additional rules and remarks}
\begin{enumerate}[nosep]
	\item \textbf{Command Generator:} Tasks will be generated using the official \emph{GPSR Command Generator} available 2 months prior to the competition in the official repository.

	\item \textbf{Naive Operators:} Optionally, commands can be issued by a \emph{Naive Operator}, i.e.~a person from the audience with no background on robotics. The referee gives the command to the \emph{Naive Operator}, who will then issue it to the robot (rephrasing is allowed). If the robot consistently fails to understand the naive operator (e.g.~3 times or more), teams can default to a custom operator.
	\\[2mm]\textbf{Remark:} Referees are not allowed to instruct naive operators on how to operate the robot. \textbf{Teams attempting to instruct or bias the operator will be disqualified}.\\[2mm]

	\item \textbf{Data Recording:} Only when using Naive Operators, a team can get an additional scoring bonus by providing the recording and transcript of the issued commands.

	\item \textbf{Deus ex Machina:} Score reduction applies per given command as follows:
	\begin{itemize}[nosep]
	\item \textbf{Custom operator:} Providing a custom operator causes 50pts score reduction.
	\item \textbf{Further assistance:} Helping a robot to accomplish a task causes 50--200pts score reduction, based on referee criterion.
	\item \textbf{Bypassing commands:} A robot instructing a human assistant on how to accomplish the whole task receives no points for the command.
	\end{itemize}

	\item \textbf{Instruction Point:} At the beginning of the test, and after finishing the first and second command, the robot moves to the \textit{Instruction Point}.

	\item \textbf{Leaving the arena:} A bonus scoring of 150pts can be earned if the robot autonomously leaves the arena after successfully executing all three given commands.
\end{enumerate}

\subsection*{OC instructions}
\textbf{2 hours before the test}
\begin{itemize}[nosep]
	\item Pre-generate and conceal commands for the robots.
	\item Announce the location of the instruction point.
	\item Recruit volunteers to assist during the test.
\end{itemize}
\textbf{During the test}
\begin{itemize}[nosep]
	\item Rearrange the arena to its normal condition.
\end{itemize}

\subsection*{Referee instructions}
\begin{itemize}[nosep]
	\item Provide the commands to the operators.
\end{itemize}


\subsection*{Score sheet}
The maximum time for this test is 5 minutes.

\begin{scorelist}
	\scoreheading{Main Goal}
	\scoreitem[3]{250}{Perform each task}
	\penaltyitem[3]{50}{Using custom operator or bypassing ASR}

	\scoreheading{Bonus rewards}
	\scoreitem[3]{100}{Understand command given by naive operator}
	\scoreitem[3]{100}{Provide audio recording and transcript}
	\scoreitem{150}{Autonomously leaving the arena}

	%\setTotalScore{1000}
\end{scorelist}


% Local Variables:
% TeX-master: "Rulebook"
% End:


% Local Variables:
% TeX-master: "Rulebook"
% End:


\newpage
\input{tests/HelpMeCarry}

\newpage
\input{tests/SPR}

\newpage
\section{Storing Groceries [DSPL \& OPL]}
The robot helps by storing newly bought groceries in the cupboard next to the objects of the same kind that are already there; for instance by placing fresh apples near other apples.

\subsection{Goal}
The robot has to correctly identify and manipulate objects at different heights, grouping them by category and likelihood.

\subsection{Focus}
This test focuses on the detection and recognition of objects and their features, as well as object manipulation.

% %% %%%%%%%%%%%%%%%%%%%%%%%%%%%%%%%%%%%%%%%%%%%%%%%%%%%%%%
%
% Setup
%
% %% %%%%%%%%%%%%%%%%%%%%%%%%%%%%%%%%%%%%%%%%%%%%%%%%%%%%%%
\begin{minipage}{0.70\textwidth}
	\subsection{Setup}
	\begin{enumerate}
		\item \textbf{Location:} This test can take place either inside or outside the arena. The testing area must have a bookcase or cupboard, and a nearby table. The maximum distance between the Table and the Cupboard is 2 meters.
		\item \textbf{Start position:} The robot starts between the cupboard and the table in a random orientation, but facing towards the Cupboard.
		\item \textbf{Cupboard:} The cupboard has 5 shelves between 0.0m and 1.80m from the ground and contains several objects grouped by category or likeliness (See \ref{rule:scenario_objects}). The cupboard has at least one free space for starting a new set.
		\begin{itemize}
		 	\item \textbf{Door:} The cupboard has a single door, which is closed initially.
		 	This door encloses some of the objects, covering up to one half of the cupboard (e.g. the left or bottom half), as indicated by the hatched area in Figure \ref{fig:storing_groceries_shelf}.
		\end{itemize}
		\item \textbf{Table:} The table has at least 5 objects (but no more than 10). If not all objects fit on the table, they will be added as the robot frees up space.
	\end{enumerate}
\end{minipage}\hfill
\begin{minipage}{0.25\textwidth}
	\begin{figure}[H]
		\centering
		\includegraphics[width=\textwidth]{images/storing_groceries.png}%
		\vspace{-10pt}
		%\caption{Example shelf where objects will be placed.}
		\caption{Shelf}
		\label{fig:storing_groceries_shelf}
	\end{figure}
\end{minipage}

% %% %%%%%%%%%%%%%%%%%%%%%%%%%%%%%%%%%%%%%%%%%%%%%%%%%%%%%%
%
% Task
%
% %% %%%%%%%%%%%%%%%%%%%%%%%%%%%%%%%%%%%%%%%%%%%%%%%%%%%%%%
\subsection{Task}
\begin{enumerate}
	\item \textbf{Evaluating the situation:} The robot inspects its surrounding and analyzing the best course of action. In any order, the robot has to:
	\begin{itemize}
		\item \textit{Inspect the cupboard} (locating and categorizing existing groceries).
		\item \textit{Open the cupboard's door.} If the robot can't open the door, it may ask the Referee to do it.
		\item \textit{Inspect the table} (analyze the newly bought groceries, i.e. objects).
	\end{itemize}

	\item \textbf{Moving objects:} The robot moves as many objects as possible in the given time
	(only the first five score)
	from the Table to the Cupboard, allocating similar objects all together.
	Stacking is allowed.
	\begin{itemize}
		\item Objects of the same type (i.e. identical known objects or akin alike objects) must be placed one next to the other.
		\item If the Cupboard has no object of the same type, then objects must be grouped by category (e.g. drinks with drinks, snacks with snacks, etc)
		\item If the Cupboard has no similar object, the robot must clearly state its decision on how to solve the problem. For instance, the robot can start a new set in a free space for either all unknown objects or all objects sharing a particular feature (color, shape, function, etc.).
		\item Moving two objects at a time (2-handed manipulation) is allowed.
	\end{itemize}

	\textbf{Note:} Either before or after grasping an object the robot may announce the name of the object found.
	\item \textbf{Repeat:} This repeats until the time is up or all groceries are stored.
\end{enumerate}



% %% %%%%%%%%%%%%%%%%%%%%%%%%%%%%%%%%%%%%%%%%%%%%%%%%%%%%%%
%
% Additional Rules
%
% %% %%%%%%%%%%%%%%%%%%%%%%%%%%%%%%%%%%%%%%%%%%%%%%%%%%%%%%
\subsection{Additional rules and remarks}
\begin{enumerate}
	\item \textbf{Bypassing Manipulation:} Bypassing object manipulation via the CONTINUE rule (Section \refsec{rule:asrcontinue}) is not allowed during this test.
	\item \textbf{No setup:} There is no setup time.
	\item \textbf{Startup:} The robot can be started with a simple voice command or via a start button (Section \refsec{rule:start_signal}).
	\item \textbf{Single try:} The robot must be able to start from the first attempt. There is no restart for this test. If the robot is unable to start it must be removed immediately.
	\item \textbf{Collisions:} Slightly touching the cupboard is tolerated (but not advised). Crushing objects or any other form of a major collision terminates the test immediately (Section \refsec{rule:safetyfirst}).
	\item \textbf{Clear area:} The robot may assume that the direct vicinity of the cupboard and table are clear, and that the robot can move slightly backwards for its task.
	\item \textbf{Objects:} The 10 objects are evenly distributed in random fashion including
	3 known objects,
	3 alike objects,
	2 unknown objects, and
	2 special objects (bowl, cloth, dish, etc.).
	\item \textbf{Table} The table's rough location will be announced beforehand, having it's position either left, right, or behind the robot.
	% \item \textbf{Timing:} The robot has to successfully place the first object within the first two minutes, otherwise the test is ended. If the robot opens the cupboard door by on its own, one additional minute is added to the 2-minutes limit. The maximum time for this test is 5 minutes.
\end{enumerate}

% \subsection{Data recording}
% Please record the following data (See \refsec{rule:datarecording}):
% \begin{itemize}
% 	\item Images
% 	\item Plans
% \end{itemize}

\subsection{OC instructions}

\textbf{2 hours before the test}
\begin{itemize}
    \item Announce the startup location for robots.
    % This test is about manipulation and object recognition, NOT about finding furniture
    % Finding a table is done in the Restaurant task
    \item Announce which table will be used in the test.
    \item Announce a rough location for the table.
\end{itemize}

\subsection{Referee instructions}
The referee needs to
\begin{itemize}
	\item Place the objects in the cupboard and a few of the same class on the table. New items can be placed when there is room or the robot asks for more objects.
	\item Close the door of the cupboard.
	\item Put objects on the table and the corresponding objects in the cupboard: 3 known objects, 2 alike and 5 unknown objects.
\end{itemize}


\newpage
\subsection{Score sheet}

The maximum time for this test is 5 minutes.

\begin{scorelist}
	\scoreheading{Main Goal}
	\scoreitem[5]{100}{Move an object next to their peers in the shelf}
	\penaltyitem[5]{-30}{Receiving human help (point at target location)}
	\penaltyitem[5]{-100}{Receiving human help (move object)}

	\scoreheading{Bonus rewards}
	\scoreitem{300}{Opening the shelf door without human help}
	\scoreitem{100}{Moving a \emph{tiny} object}
	\scoreitem{100}{Moving a \emph{heavy} object}

	% No longer necessary, computes automatically
	% \setTotalScore{1000}
\end{scorelist}


% Local Variables:
% TeX-master: "Rulebook"
% End:


% Local Variables:
% TeX-master: "Rulebook"
% End:


\chapter{Tests in Stage II}
\label{chap:stage_II}

\begin{itshape}
All ability and integration tests in \iterm{Stage~II}  are performed only once. Some tests have optional tasks that grant additional points when performed correctly, clean and fast. The \iaterm{Technical Committee}{TC} must be informed if a team is planning to perform any of the optional tasks. Unless explicitly stated otherwise, no additional time is given while performing optional tasks.

In the \iterm{Open Challenge} the robot must be able to show to the \iaterm{Technical Committee}{TC} the achievements on the main research line of its own team. This test may grant up to 250 points, never exceeding the maximum scoring achieved in Stage II.

\section*{Robot \& team cooperation}
We encourage robots and teams to work together when performing tests.
For scoring, points are awarded per subtask. The robot (and thus team) performing the subtask gets the points.
For example, in the Restaurant test, if one robot of team A can take the order and another robot of team B delivers the order, then the points for taking the order go to team A, while the points for delivering go to team B.
Of course, team A \& B can both perform the test in their own turn.

\end{itshape}

\newpage
%%%%%%%%%%%%%%%%%%%%%%%%%%%%%%%%%%%%%%%%%%%%%%%%%%%%%%%%%%%%%%%%%%%%%%%%%%%%%
%
% EEGPSR
%
%%%%%%%%%%%%%%%%%%%%%%%%%%%%%%%%%%%%%%%%%%%%%%%%%%%%%%%%%%%%%%%%%%%%%%%%%%%%%

% Number of concurrent teams
\newcommand{\eegpsrTeams}{2~}
% Maximum number of commands to be given to a robot
\newcommand{\eegpsrMaxCmd}{3~}
% Maximum amount of time given to a team to perform a single command
\newcommand{\eegpsrMaxCmdTime}{5~}
% Maximum amount of time given to a team to perform all commands
\newcommand{\eegpsrMaxTeamTime}{\eegpsrMaxCmd$\times$\eegpsrMaxCmdTime}

% \section[EEGPSR]{E\textsuperscript{2}GPSR \\ \normalsize{(Enhanced Endurance General Purpose Service Robot)}}
\section[EEGPSR]{Enhanced Endurance General Purpose Service Robot}
\label{sec:eegpsr}

%
% MAURICIO @2017
% Short instructions based on GPSR
%
This test evaluates the abilities of the robot that are required throughout the set of tests in Stage I and stage II of this and previous years' RuleBooks. In this test the robot has to solve multiple tasks upon request over an extended period of time (30-45 minutes). That is, the test is not incorporated into a (predefined) story and there is neither a predefined order of tasks nor a predefined set of actions. The actions that are to be carried out by the robot are chosen randomly by the referees from a larger set of actions. These actions are organized in several categories targeting an special ability. Scoring depends on the abilities shown.

\subsection{Focus}
This test particularly focuses on the following aspects:
\begin{itemize}
	\item No predefined order of actions to carry out (to get away from state machine-like behavior programming).
	\item Increased complexity in speech recognition.
	\item Environmental (high-level) reasoning.
	\item Robust long-term operation.
\end{itemize}


\subsection{Task}

\begin{enumerate}
	\item \textbf{Entering and command retrieval:} The robot enters the arena and drives to a designated position where it has to wait for further commands. \\

	\item \textbf{Command generation:} A command is generated randomly, depending on the command category chosen by the team (see below). \\

	\begin{enumerate}
		\item \textbf{Category I: Three at once.} The command is composed by \textit{three simple actions}, which the robot has to show it has recognized. the actions are much like the ones of GPSR. The robot may repeat the understood command and ask for confirmation. If it can't recognize the command correctly, it can also ask the speaker to repeat the complete command.

		\item \textbf{Category II: People.} The given commands focuses in interacting with people. Tasks in this category involve following or guiding people inside and outside the arena, recognize people's gestures or a specific person given its description, and remembering previously known people.

		\item \textbf{Category III: Objects.} The given commands focuses interacting with objects. Tasks in this category involve setting up a table, describe the objects placed on a table or shelf, and deliver objects that match a description or are located inside a cupboard or drawer.
	\end{enumerate}
		
	The robot can work on at most \eegpsrMaxCmd commands within each of the following scenarios randomly chosen by the referee: \\

	\begin{itemize}
		\item \textbf{Complete command.} The robot gets a command containing all the information required for its execution.

		\item \textbf{Incomplete command.} The robot gets a command that does not include all the information necessary to accomplish the task. The robot may either request the missing information (by asking reasonable questions), or attempt to solve the command on its own.

		\item \textbf{Erroneous or misleading command.} The command contains erroneous misleading information. The robot should be able to realize what went wrong and come up with a solution. In addition, it must go back to the operator and clearly state \textit{what} went wrong and \textit{how} it was fixed, or \textit{why} it wasn't able to accomplish the task.
	\end{itemize}

	\item \textbf{Task assignment:} The robot is given the command by the operator and may directly start to work on the task assignment.

	\item \textbf{Task execution:} The robot must stop the execution of a task and return to its designated position within \eegpsrMaxCmdTime minutes. Otherwise the robot must be moved to its designated position immediately. If a restart is still available to the team, it can be restarted at the designated position. \\

	\item \textbf{Returning:} After accomplishing the assigned task, the robot has to move back to its designated position to wait and retrieve the next command (i.e., go back to 1. without the need of re-entering the arena). \\

	\item \textbf{Timing:} The total time allotted to the robot for command retrieval and task execution is \eegpsrMaxTeamTime minutes. If the robot is not at its designated position after the time has expired, it must be moved at its designated position immediately.\\

\end{enumerate}

\subsection{Additional rules and remarks}
\label{sec:eegpsr-remarks}
\begin{enumerate}
	\item \textbf{CONTINUE rule:} Teams are able to use the CONTINUE rule in this test, with all the standard penalties it involves as described in section \refsec{rule:continue}.
	%The CONTINUE rule can only be used with the custom operator (e.g. both penalties of custom speaker and CONTINUE rule will be applied). 
	\\

	\item \textbf{Number of Teams and Scheduling:} In each test slot, \eegpsrTeams teams may be competing in the arena concurrently. The robots will be tested in an interleaved fashion: The robots will retrieve commands and execute the task one after the other. As stated above, each robot will have a maximum amount of \eegpsrMaxCmdTime minutes per command (including time for retrieving the command and executing it). \\
	
	\item \textbf{Returning to designated position:} To facilitate a fluent and untroubled performance of the robots, they must return (or being returned) to their designated position before the \eegpsrMaxCmdTime minutes command time elapses. \textbf{If a robot moves from its designated position while another robot is working on a command, it must be immediately disabled} and moved to its designated position. If a restart is still available to the team, it can be restarted at its designated position. \\

	\item \textbf{Referees:} Since the score system in this test involves a subjective evaluation of the robot's behavior, the referees are EC/TC members. One referee is assigned to each team to judge performance, to measure the time for working on a command, and to keep track of the overall operating time of the robot. \\

	\item \textbf{Category selection:} For every of the three commands given to the robot, the team chooses the desired command category.\\

	\item \textbf{Operator:}
	\begin{itemize}
		\item The person operating the robot is one of the referees (default operator).
		\item If the robot appears to consistently not be able to understand the operator, the referees ask the team to apply the CONTINUE rule (\refsec{rule:asrcontinue}).
	\end{itemize}

	\item \textbf{Inoperative robots:} If a robot gets stuck while trying to accomplish a task during a reasonable amount of time (e.g.~30 seconds), the referee may ask the team to move back the robot to its designated position, proceeding with the next robot. \\

	\item \textbf{Restart:} Robots will be restarted at their designated position (starting outside the arena is prohibited). If a robot requires a restart, the referee will proceed with the next robot.\\

	\item \textbf{Changing/Charging batteries:} The team may install a charging station at the designated position of the robot, if it does not hinder the other robots. However, the robot must connect itself with the charging station after carrying out a command. Changing batteries or manually connecting the robot with the charging station is allowed during a restart. \\

	\item \textbf{Scoring:} Robots are scored by successfully performed ability and full command completion within time. 
\end{enumerate}

\subsection{OC instructions}
\textbf{2h before test:}
\begin{itemize}
	\item Specify and announce the entrance/exit door for each robot. 
	\item Specify and announce the waiting position for each robot. 
\end{itemize}
\textbf{During the test:}
\begin{itemize}
	\item Help placing items and arranging people upon referee request.
\end{itemize}

\subsection{Referee instructions}
\textbf{During the test:}
\begin{itemize}
	\item Generate random sentences. %by an automatic sentence generator.
	\item Take the command and total time per team.
\end{itemize}


\newpage
\subsection{Score sheet}
The maximum time for this test is 40 minutes.

{\footnotesize

\begin{scorelist}
	\scoreheading{Performance}
	\scoreitem[3]{15}{Understanding the command the $1^{st}$ attempt}
    \scoreitem[3]{-5}{Each additional attempt (max 3)}
	\scoreitem[3]{15}{Solved random category / Each extra category (mix)}
	
	\scoreheading{HRI}
	\scoreitem{ 5}{Answering a predefined question / fetching orders after event detection}
	\scoreitem{10}{Missing information retrieval / Providing explanations}
	\scoreitem{20}{Natural handover (give or take)}

	\scoreheading{Manipulation}
	\scoreitem{ 5}{Grab/place an object}
	\scoreitem{30}{Manipulation: two-handed, in narrow spaces*, tiny/heavy/slippery* objects}
	\scoreitem{20}{Open/close a door/drawer*}
	\scoreitem{50}{Pouring, pressing, uncapping, turning on/off, twisting}
	
	\scoreheading{Memory \& Awareness}
	\scoreitem{10}{Detect an expected event (within a reasonable amount of time)*}
	\scoreitem{20}{Detect an unexpected event*}
	\scoreitem{20}{Give information of environment changes/given commands*}

	\scoreheading{Navigation}
	\scoreitem{10}{Follow/guide operator (inside only)}
	\scoreitem{15}{Follow/guide operator (outside)}
	
	\scoreheading{Object recognition}
	\scoreitem{ 5}{Counting all objects, recognize known/alike object}
	\scoreitem{15}{Counting objects in category / matching description*}
	\scoreitem{30}{Describe unknown objects}
	\scoreitem{30}{Find: from description*, occluded ($>50\%$)}
	\scoreitem{50}{Find hidden objects, infer category from features}

	\scoreheading{People, pose and activity recognition}
	\scoreitem{15}{Gesture detection}
	\scoreitem{15}{Finding or recognizing people}
	\scoreitem{20}{Person's geneder/pose* recognition}
	\scoreitem{15}{State the number of people in a group}

	\setTotalScore{500}
\end{scorelist}
}

% Local Variables:
% TeX-master: "Rulebook"
% End:


% Local Variables:
% TeX-master: "Rulebook"
% End:
 


\newpage
\newcommand{\bonusRobotCoop}{50~}

\section{Open Challenge}
\label{sec:test_open_challenge}

During the Open Challenge teams are encouraged to demonstrate recent research results and the best of the robots' abilities. It focuses on the demonstration of new approaches/applications, human-robot interaction and scientific value.

\subsection{Task}

The Open Challenge consists of a demonstration and an interview part.
It is an open demonstration which means that the teams may demonstrate anything they like.
The performance of the teams is evaluated by a jury consisting of all team leaders, TC and EC.
\OpenDemonstrationTask{seven}{three}

\subsection{Presentation}
During the demonstration, the team can present the addressed problem and the demonstrated approach.
\begin{itemize}
	\item A video projector or screen, if available, may be used to present a brief (max. 2 minute) presentation relevant to the demonstration.
	\item Teams may omit the video, use a more brief video, or have the robot act over the video in order to make more time for the robot demo.
	\item There may be no human presenter. This is intended to be a demonstration of the robot's capabilities and not a research talk. The robot may present for itself (e.g., describing what it is doing or providing a narrative for the presentation on its own).
	\item Humans may interact with the robot during the interaction, but are not to act as presenters. This judgement is left to the jury.
	\item The team can also visualize robot's internals, e.g., percepts.
\end{itemize}

It is important to note that the jury may decide to end the demonstration if there is nothing happening or nothing \emph{new} is happening.

\OpenDemonstrationChanges

\subsection{Jury evaluation}
\label{sec:test_open_challenge:scoring}
\begin{enumerate}
	\item \textbf{Jury of team leaders:} All teams have to provide \emph{one} person
	(preferably the team-leader) to follow and evaluate the entire Open Challenge.
	\item \textbf{Evaluation:} Both the demonstration of the robot(s), and the answers of the team in the interview part are evaluated.\\
	For each of the following \emph{evaluation criteria}, each jury member submits a score from $0-100$:
	\begin{enumerate}
	\item Novelty and (scientific) contribution
	\item Difficulty level of the demonstrated task
	\item Success of the demonstration
	\item Overall (demo was convincing, fluent, interesting, etc.)
	\end{enumerate}
	A jury member is not allowed to evaluate and give points for the own team.
	\item \textbf{Normalization and outliers}:
	\begin{enumerate}
		\item The points given by each jury member are scaled to obtain a score from $0.0-1.0$.
		\item The normalized total score for each team is the mean of the jury member scores.
			To neglect outliers, the $N$ best and worst scores are left out:
			$$\mbox{score}_{norm} = \frac{\sum\mbox{team-leader-score}}{\mbox{number-of-teams} - (2N+1)}\times\frac{1}{100},
			\quad N=\begin{cases}2, & \mbox{number-of-teams} \ge 10\\1, & \mbox{number-of-teams} < 10 \end{cases}$$
		\end{enumerate}
		\item The final Open Challenge score for each team is computed at the end of Stage 2. The Open Challenge \scoring{final score} is the product of the normalized score multipled by the highest score achieved in Stage 2:
		$$\mbox{score} = \mbox{score}_{norm} \times \frac{min\Big(250, max\big(\{S_2\}\big)\Big)}{250},
		\quad \{S_2\}=\mbox{All Stage2 scores}
		$$
\end{enumerate}

\subsection{Additional rules and remarks}
\begin{enumerate}
	\item \textbf{Start signal:} There is no standard start-signal for this test.
	\item \textbf{Abort on request:} At any time during the demonstration, the jury may interrupt and abort the demonstration:
	\begin{enumerate}
		\item if nothing is shown: in case of longer delays (more than one minute), e.g., when the robot does not start or when it got stuck;
		\item if nothing new is shown: the demonstrated abilities were already shown in previous tests (to avoid dull demonstrations and push teams to present novel ideas).
	\end{enumerate}
\end{enumerate}

\subsubsection{Team-team-interaction:}
\label{rule:OC-team-team-interaction}
An extra bonus of up to \bonusRobotCoop points can be earned if robots from two teams (4 robots maximum, 2 from each team) successfully collaborate (robot-robot interaction).
\begin{enumerate}
	\item This bonus is earned for both teams.
	\item The robot(s) of the other team must only play a minor role in the total demonstration.
	\item It must be made clear that the demonstrations from the two teams are not similar, otherwise the points cannot be awarded.
	\item In case a team receives two (or more) bonuses, the maximum bonus will be taken.
	\item The collaboration is possible even if one of the two teams has not reached Stage 2.
	\item A team not participating in Stage 2 receives no bonus points for this test.
\end{enumerate}

\paragraph*{Inter-league collaboration}:
\label{rule:OC-inter-league-collaboration}
Inter-league collaboration must be announced to the OC at least one day before the test. Teams participating in multiple @Home Leagues does receive no bonus for cooperation. Standard Platform robots are allowed to take part in the Open Challenge of the Open Platform League, but Open Platform robots can \emph{not} participate in any Standard Platform League's test. In the same sense, DSPL robots are not allowed in SSPL and vice versa.

For sake of clarity, please consider the following example: Let be A, B two teams participating in RoboCup @Home where
\begin{itemize}
	\item Team A participates in SSPL.
	\item Team B participates in both SSPL and OPL.
	\item Team A and B have qualified into Stage II.
\end{itemize}
Then, by applying the \textit{Inter-league collaboration Rule} (See \refsec{rule:OC-inter-league-collaboration}) the following statements can be concluded:
\begin{itemize}
	\item B OPL can not participate in A SSPL's open challenge.
	\item B OPL can not participate in B SSPL's open challenge.
	\item A SSPL can participate in B OPL's open challenge. Team A and B get a bonus because A <> B.
	\item B SSPL can participate in B OPL's open challenge. There is no bonus because B = B.
\end{itemize}




% Local Variables:
% TeX-master: "Rulebook"
% End:


\newpage
\section{Procter \& Gamble Dishwasher Challenge [DSPL \& OPL]}

\newcommand{\openpart}{%
All teams are allowed to participate and compete in the \textit{Procter \& Gamble Challenge} regardless of whether they advanced to the Stage II or not, and get the award.
}

The robot has to remove all dishes from a table (presumably after dinner) and place them into the dishwasher.

\subsection{Open Participation}
\openpart

\subsection{Focus}
This test focuses on object perception, manipulation, and planning.

\subsection{Setup}
\begin{itemize}
	\item \textbf{Location:} This test takes place in the arena. A dining table is located close to the dishwasher.
	\item \textbf{Dishwasher:} The dishwasher is near the table, preferably located in the same room. The dish washer is open and with all racks out.
	\item \textbf{Tray:} A plastic tray is located either on top of the dishwasher, or onto one of its racks. The tray may have tableware and cutlery placed inside already.
	\item \textbf{Table setting:} The table has several objects disposed in a typical setting for a meal for one person. These objects include tableware (e.g. place mats, napkins, dishes, glasses), and silverware (e.g. forks, spoons, knives).
	\item \textbf{Spot:} There is a dirty spot on the table next to the table setting that requires cleaning.
\end{itemize}

\subsection{Task}
	\begin{enumerate}
		\item \textbf{Entering the arena:} The robot enters the arena and navigates to the designated location.
		% \item \textbf{Fetching command:} The operator requests the robot to clean up the table.
		\item \textbf{Clean the table:} The robot takes all the tableware and cutlery to either the dishwasher or to the tray, as instructed (team's choice).
		\item \textbf{Filling the dishwasher:} If the robot placed the objects into the tray, it must proceed to put the tray onto one of the dishwasher racks.
		\item \textbf{Place the Cascade Pod:} The robot places the  Cascade Pod~into the dishwasher, preferrably inside the tab compartment.
		\item \textbf{Scrub spots and spills:} The robot detects spots and spills on the table and clean them up using the cleaning cloth or sponge it has retrieved previously.
		\item \textbf{Leave the arena:} The robot leaves the arena once it has finished cleaning.
	\end{enumerate}

\subsection{Additional rules and remarks}
\begin{enumerate}
	\item \textbf{Collisions:} Slightly touching the table is allowed, as well as slightly pushing some objects. However, driving over the objects or any other form of a major collision is not allowed, and the referees directly stop the robot (Section \refsec{rule:safetyfirst}).

	\item \textbf{Objects:} A total of 6 objects are used in this test following the distribution shown below:
	\begin{itemize}
		\item\textit{Silverware}: Any two objects.
		\item\textit{Tableware}: Any three objects, excluding silverware. At least one must be a dish.
		\item\textit{Cascade Pod}: One Cascade Pod.
	\end{itemize}
	All objects used in this test are taken from the list of standard objects (See \ref{rule:scenario_objects}). All of them are considered to be known to the robot.

	\item \textbf{Safe placing:} Objects placed in the rack or tray must be placed gently and safely. It must be clear to the referee that the robot is trying to put the object in place and not dumping, throwing, or dropping it. Dumped and dropped objects won't be scored even if they land in the dishwasher/tray.

	\item \textbf{Special Objects:} The plastic tray and the Cascade Pod are provided by \textit{Procter \& Gamble} and are considered special objects. Teams are not allowed to use substitutes of them in this test.

	\item \textbf{Spots and Spills:} The referees must place a spot (e.g. jam or chocolate syrup) or spill some liquid (e.g.milk) on the table before the test starts. The substance used to create the spot shall be clearly visible and contrasting with the table. When cleaning it, it must be clear the robot has detected the spot and is trying actively to clean it. The selection of the cleaning tool (sponge or cloth) is made by the team.\\
	\textbf{Remark:} When possible, no tablecloth will be used to ease cleaning. If removing the tablecloth is not possible, a dry spot will be used instead (e.g. breadcrumbs or coffee powder).

	\item \textbf{Dishwasher:} Is up to the team to decide whether the robot will place the objects in the dishwasher's rack or in the official tray. When using the tray, it should be loaded into the dishwasher.

	\item \textbf{Dishwasher door:} Unless requested otherwise by the team, the dishwasher is open and with the racks pulled out by default. The team leader can, however, request the diswasher to be closed and score additional points for opening it. If the robot fails to open the door, it must clearly state it and request the referee to open it.

	\item \textbf{Human-Robot Interaction:} The robot is allowed to
	\begin{enumerate*}[label={\alph*)}]
	\item indicate the location of the cloth or sponge,
	\item ask the human operator what to do with food leftovers,
	and
	\item request operator's help to find the spot (e.g. pointing at it).
	\end{enumerate*}
	This interaction is extensible to any kind of reasonable request from the robot when attempting to solve the task.

	\item \textbf{Open Participation}
	\openpart
	However, scoring in this test will be only considered for those teams who have advanced to Stage II. This way, no Stage I team can have an overall score higher than a Stage II team.

\end{enumerate}

% \subsection{Data recording}
% Please record the following data (See \refsec{rule:datarecording}):
% \begin{itemize}
% 	\item Images of recognized objects
% 	\item List of moved items
% \end{itemize}

\subsection{Referee instructions}

The referee needs to
\begin{itemize}
	\item Place the table setting.
	\item Clean spots smudged by the previous robot.
	\item Place the new spot meant to be clean.
	\item Place the tray on the dishwasher or onto the rack, as requested by the team.
\end{itemize}

\subsection{OC instructions}
During Setup days:
\begin{itemize}
	\item Provide official cutlery and tableware for training.
\end{itemize}

2 hours before the test:
\begin{itemize}
	\item Announce the predefined location to take the command.
	\item Announce the predefined location of the Cascade Pod.
\end{itemize}

\newpage
\subsection{Score sheet}
The maximum time for this test is \textbf{10 minutes}.

\begin{scorelist}
	\scoreheading{Opening the dishwasher} %50 pts
	\scoreitem{50}{Autonomously opening the dishwasher}

	\scoreheading{Filling the dishwasher (direct)} %200 pts
	\scoreitem[3]{40}{Safely placing a tableware item in the dishwasher's rack}
	\scoreitem[2]{40}{Safely placing a cutlery item in the dishwasher's basket}

	\scoreheading{Filling the dishwasher (tray)} %200 pts
	\scoreitem[3]{30}{Safely placing a tableware item in the tray}
	\scoreitem[2]{35}{Safely placing a cutlery item in the tray}
	\scoreitem{40}{Placing the tray into the dishwasher}

	\scoreheading{Placing the cascade-pod} %40 pts
	\scoreitem{40}{Placing the cascade-pod in the dishwasher's soap compartment}
	\scoreitem{20}{Placing the cascade-pod in the dishwasher (somewhere else)}

	\scoreheading{Cleaning the table} %50 pts
	\scoreitem{50}{Successfully cleaning the spot}
	\scoreitem[-1]{20}{Receiving operator's assistance to find the spot}
	\scoreitem[-1]{20}{Smudging the spot while trying to clean it}

	\scoreheading{Leave the arena} %10 pts
	\scoreitem{10}{Autonomously leave the arena before the time elapses}

	\setTotalScore{350}
\end{scorelist}

% Local Variables:
% TeX-master: "Rulebook"
% End:


% Local Variables:
% TeX-master: "Rulebook"
% End:



\newpage
The maximum time for this test is 15 minutes.

\small\begin{scorelist}
	\scoreheading{Main Goal}
	\scoreitem{500}{Take and serve an order}
	\penaltyitem[2]{200}{Being guided to the \emph{Kitchen-bar} or to a customer's table}
	\penaltyitem{100}{Not making eye-contact when taking an order}
	\penaltyitem[4]{50}{Bypassing object manipulation (handover)}
	\penaltyitem[5]{100}{Bypassing object manipulation}

	\scoreheading{Second order (bonus reward)}
	\scoreitem{500}{Take and serve an additional order}
	\penaltyitem[2]{200}{Being guided to the \emph{Kitchen-bar} or to a customer's table}
	\penaltyitem{100}{Not making eye-contact when taking an order}
	\penaltyitem[4]{50}{Bypassing object manipulation (handover)}
	\penaltyitem[5]{100}{Bypassing object manipulation}

	\scoreheading{Additional bonus rewards}
	\scoreitem[2]{100}{Detect calling or waving customer}
	\scoreitem[2]{100}{Reach a customer's table without prior guidance/training}
	\scoreitem[2]{300}{Use an unattached tray to transport an order}
\end{scorelist}

% Local Variables:
% TeX-master: "Rulebook"
% End:


\newpage
The maximum time for this test is \textbf{10 minutes}.

\begin{scorelist}
	\scoreheading{Engaging spectators} % Max 50
	\scoreitem{30}{Find an spectator (or group)}
	\scoreitem{20}{Greet an spectator (handshake)}
	\scoreitem{10}{Greet and get greet by an spectator (bowing or waving)}

	\scoreheading{Guiding spectators} % Max 50
	\scoreitem{10}{Convince spectator to follow}
	\scoreitem{40}{Reach the audience area}

	\scoreheading{Q\&A Session} % Max 210
	% \scoreitem{10}{Finish talk without loosing spectators attention}
	\scoreitem{10}{Finish talk without loosing spectators}
	\scoreitem[2]{70}{Each correctly understood question}
	\scoreitem[2]{30}{Each correctly answered question}
	
	\scoreheading{Bilingual interaction} % Max 80
	\scoreitem{10}{Bilingual engaging}
	\scoreitem[2]{25}{Questions in $2^{nd}$ language}
	\scoreitem[2]{10}{Question answered also in $2^{nd}$ language}
	
	\setTotalScore{390}
\end{scorelist}


% Local Variables:
% TeX-master: "Rulebook"
% End:


\newpage
\chapter{Finals}

The competition ends with the Finals on the last day, where the four teams with the highest total score compete.
The \iterm{Finals} are conducted as a final open demonstration.
This demonstration does not have to be different from the Open Challenge. 
It does not have to be the same either.

To avoid logistical issues during the last day of the competition, the \iterm{Finals} are divided into two sets of demonstrations: the Bronze Competition and the RoboCup @Home Grand Finale.
The Bronze Competition is a set of demonstrations that are carried out before the RoboCup @home Grand Finale. Here, all the leagues run in parallel, with the fourth and third highest scored teams competing for the bronze.
Finally, the two teams with the highest score in each League present their demonstrations in a serialized manner during the RoboCup @Home Grand Finale.

Even though each league has its own first, second and third place, the RoboCup @Home Grand Finale is meant to show the best of all leagues to the jury members as well as the audience and, thus, warrants a single schedule slot.

\section{Evaluating Juries for Final Demonstrations}
\label{final:jury}
Each set of final demonstrations is evaluated by a different combination of evaluating juries, here described.

\begin{enumerate}
\item\textbf{League-internal jury:} The league-internal jury is formed by the Executive Committee.
The evaluation of the league-internal jury is based on the following criteria:
  \begin{compactenum}
  \item Scientific contribution %(maybe taken from the OC)
  \item Contribution to @Home %(evaluated by Execs/TC)
  \item Relevance for @Home / Novelty of approaches %(evaluated by execs/TC)
  \item Presentation and performance in the finals.
  \end{compactenum}

\item \textbf{League-external jury:} The league-external jury consists of people not being involved in the RoboCup@Home league,
but having a related background (not necessarily robotics).
They are appointed by the Executive Committee.
The evaluation of the league-external jury is based on the following criteria:
  \begin{compactenum}
  \item Originality and Presentation
    (story-telling is to be rewarded)
  \item Usability / Human-robot interaction
  \item Multi-modality / System integration
  \item Difficulty and success of the performance
  \item Relevance / Usefulness for daily life
  \end{compactenum}

\item\textbf{Teams-based jury:} The teams-based jury is formed by members of the league's teams.
The evaluation of the teams-based jury is based on the following criteria:
  \begin{compactenum}
  \item Scientific contribution %(maybe taken from the OC)
  \item Contribution to @Home %(evaluated by Execs/TC)
  \item Relevance for @Home / Novelty of approaches %(evaluated by execs/TC)
  \item Presentation and performance in the finals.
  \end{compactenum}
\end{enumerate}


\section{Bronze Competition (4th and 3rd Highest Scoring Teams)}
The demonstration is evaluated by one member of the league-internal jury, by one member of the league-external jury and by the complete team-based jury.
The final score and ranking are determined by the jury evaluations and by the previous performance (in Stages I and II) of the team, in the following manner:

\begin{enumerate}
  \item The influence of the league-internal jury member to the final ranking is \SI{15}{\percent}.
  \item The influence of the league-external jury member to the final ranking is \SI{15}{\percent}.
  \item The influence of the teams-based jury to the final ranking is \SI{15}{\percent}.
  \item The influence of the total sum of points scored by the team in Stage I and II is \SI{55}{\percent}.
\end{enumerate}

These demonstrations are carried out in parallel, having each League perform their own Bronze Competition in their own arena at the same time to save time.

\section{RoboCup@Home Grand Finale (2nd and 1st Highest Scoring Teams)}
The demonstration is evaluated by the complete league-internal and the complete league-external jury.
The final score and ranking are determined by the jury evaluations and by the previous performance (in Stages I and II) of the team, in the following manner:
  
\begin{enumerate}
  \item The influence of the league-internal jury to the final ranking is \SI{25}{\percent}.
  \item The influence of the league-external jury to the final ranking is \SI{25}{\percent}.
  \item The influence of the total sum of points scored by the team in Stage I and II is \SI{50}{\percent}.
\end{enumerate}

These demonstrations are carried out in a serialized fashion, one League performing after another in one arena.


\section{Common Description of Final Demonstrations}
Teams can choose freely what to demonstrate, however it is expected that teams present the scientific and technical contributions they submitted in both \iterm{team description paper} and the \iterm{RoboCup\char64Home Wiki}.
In addition, teams may provide a printed document to the jury (max 2 pages) that summarizes the demonstrated robot capabilities and contributions.  

\subsection{Task}
The procedure for the demonstration and the timing of slots is as follows:
\OpenDemonstrationTask{ten}{five}

\OpenDemonstrationChanges

%% %%%%%%%%%%%%%%%%%%%%%%%%
\section{Final Ranking and Winner}

The winner of the competition is the team that gets the highest
ranking in the finals.

There will be an award for 1st, 2nd and 3rd place. All teams in the
Finals receive a certificate stating that they made it into the Finals
of the RoboCup@Home competition.


% Local Variables:
% TeX-master: "Rulebook"
% End:


\begin{appendices}
% \addto\captionsenglish{\renewcommand{\chaptername}{Appendix}}
% \renewcommand{\chaptername}{Appendix}
\renewcommand*{\chapterformat}{\LARGE{Appendix \thechapter}}
\renewcommand{\chaptermark}[1]{\markboth{\appendixname \ \thechapter. \ #1}{}}

% \input{RoboNurse-Diseases}

\input{ExampleSkills}
\chapter[EEGPSR in detail]{E\textsuperscript{2}GPSR in detail.}
\label{chap:eegpsr-appendix}

\section{Command Generation}
EEGPSR commands are generated randomly using the official [EE]GPSR Command Generator and grammars publicly available at https://github.com/kyordhel/GPSRCmdGen. The official [EE]GPSR Command Generator and the official grammars will be made available two months before the competition. However, teams must be aware that the categories, objects and other data is provided for testing purposes only.

For each command to be executed, the Team Leader must choose a Command Category. If the Team Leader knows \textit{a priori} that the robot won't be able to execute the generated command, is advised to inform the operator immediately in order to proceed with the next command, saving this way valuable time for the task execution.

\subsection{Random Category Selection}
The team leader may request \textbf{once} to the referee to give to the robot a command from a random category. Extra points are given if the robot is able to successfully execute the given command. The random category selection is a \textit{one-time} request.

\subsection{Mixing Categories}
The team leader may request to the Technical Committee to test the robot with commands involving abilities from two or more categories. Mixing categories must be requested to the TC two hours before the test, and once requested there is no step back. Extra points are given if the robot is able to successfully execute the given command.


\section{Command retrieval explained}
The robot has to show it has understood the given command by stating all the required information to accomplish the task. For this purpose, the robot may repeat the understood command and ask for confirmation. It is not required to repeat the command word by word; rephrasing the command is allowed. For instance, if the robot is instructed to \quotes{place a coke onto the tray}, the robot may either say: \textit{\quotes{You want me to place a coke on the tray. Is that correct?}} or \textit{\quotes{do you want me to deliver a coke to the tray?}}.

If The robot can't correctly recognize the given command, it is allowed to request the operator to repeat the command up to three times. After three failed attempts, a new command is generated. The team may opt to use a custom operator or bypassing speech recognition (\refsec{rule:asrcontinue}) at any time, but each generated command will be given to the robot no more than three times. Only three different commands are generated for a robot, if the robot fails to recognize all three commands (i.e. nine attempts), the test ends immediately.

When a robot has partially understood the command, it is allowed to ask the operator for additional information (e.g. \textit{\quotes{did you say apple juice or pineapple juice?}}).

%%%%%%%%%%%%%%%%%%%%%%%%%%%%%%%%%%%%%%%%%%%%%%%%%%%%%%%%%%%%%%%%%%%%%%%%%%%%%
%
% Categories explained
%
%%%%%%%%%%%%%%%%%%%%%%%%%%%%%%%%%%%%%%%%%%%%%%%%%%%%%%%%%%%%%%%%%%%%%%%%%%%%%
\section{Categories explained}
\label{sec:eegpsr-categories-explained}
This section explain each of the categories of the test and provides examples on how the abilities are scored.

It is important to remark that there is no script or predefined way to solve the tasks, being most of them of ambiguous nature. It is up to the team to choose how to solve each tasks accordingly with the robot's capabilities.


%%%%%%%%%%%%%%%%%%%%%%%%%%%%%%%%%%%%%%%%%%%%%%%%%%%%%%%%%%%%%%%%%%%%%%%%%%%%%
%
% Category I explained
%
%%%%%%%%%%%%%%%%%%%%%%%%%%%%%%%%%%%%%%%%%%%%%%%%%%%%%%%%%%%%%%%%%%%%%%%%%%%%%
\subsection{Category I: Advanced Manipulation}
\label{sec:eegpsr-category1-explained}
Tasks from this category require handling objects into small or narrow spaces, manipulate tools, buttons, panels, and doors; two-handed manipulation, or eye-hand coordination. 

\subsubsection{Task examples}
\begin{itemize}
	\item Grasping objects from a box.
	\item Placing objects into a microwave.
	\item Shutdown the TV using its remote control.
	\item Transporting a tray.
	\item Pouring cereal in a bowl.
	\item Opening a bottle (twist, uncap, etc.).
\end{itemize}

\subsubsection{Command examples}
\begin{itemize}
	\item Hand me a coke from the fridge (the coke is inside the fridge).
	\item Bring me some flakes in a bowl.
	\item Put this book into the drawer.
	\item Turn off the TV.
	\item Put all the beverages on the dinner table.
\end{itemize}



%%%%%%%%%%%%%%%%%%%%%%%%%%%%%%%%%%%%%%%%%%%%%%%%%%%%%%%%%%%%%%%%%%%%%%%%%%%%%
%
% Category II explained
%
%%%%%%%%%%%%%%%%%%%%%%%%%%%%%%%%%%%%%%%%%%%%%%%%%%%%%%%%%%%%%%%%%%%%%%%%%%%%%
\subsection{Category II: Advanced Object Recognition}
\label{sec:eegpsr-category2-explained}
Tasks from this category require describing unknown objects, recognize objects from description, identify occluded objects and from the distance.

\subsubsection{Task examples}
\begin{itemize}
	\item Counting objects in a shelf.
	\item Describing unknown objects.
	\item Finding object from far distance.
	\item Finding objects from a description.
	\item Infer unknown object's class (category, e.g. snacks) from features.
	\item Object detection and recognition of occluded or hidden objects (behind of, inside of, etc.).
\end{itemize}

\subsubsection{Command examples}
\begin{itemize}
	\item Bring me the biggest pill bottle from the kitchen counter.
	\item Bring me the bookcase's right-most object.
	\item Describe the objects on the drawer to me.
	\item Tell me how many red apples are in the basket on the kitchen table.
	\item Count the snacks in the shelf and tell me how many there are.
\end{itemize}


%%%%%%%%%%%%%%%%%%%%%%%%%%%%%%%%%%%%%%%%%%%%%%%%%%%%%%%%%%%%%%%%%%%%%%%%%%%%%
%
% Category III explained
%
%%%%%%%%%%%%%%%%%%%%%%%%%%%%%%%%%%%%%%%%%%%%%%%%%%%%%%%%%%%%%%%%%%%%%%%%%%%%%
\subsection{Category III: Navigation \& People Tracking}
\label{sec:eegpsr-category3-explained}
Tasks from this category require following or guiding people in crowded environments or through narrow spaces. The navigation may take place either inside or outside the arena.

\subsubsection{Task examples}
\begin{itemize}
	\item Following a person inside an elevator.
	\item Guiding a person to the toilet.
	\item Going through a multitude while following or guiding a person without loosing them.
	\item Avoiding people crossing or standing by while guiding or following.
	\item Performing real time mapping and localization.
\end{itemize}

\subsubsection{Command examples}
\begin{itemize}
	\item Guide the person at the entrance to the kitchen.
	\item 
	\item Follow the person in front of you and go to the bedroom (operator will guide the robot outside the arena, so it will need to go back). 
\end{itemize}


%%%%%%%%%%%%%%%%%%%%%%%%%%%%%%%%%%%%%%%%%%%%%%%%%%%%%%%%%%%%%%%%%%%%%%%%%%%%%
%
% Category IV explained
%
%%%%%%%%%%%%%%%%%%%%%%%%%%%%%%%%%%%%%%%%%%%%%%%%%%%%%%%%%%%%%%%%%%%%%%%%%%%%%
\subsection{Category IV: People \& Activity Recognition}
\label{sec:eegpsr-category4-explained}
Tasks from this category require memorizing a person's features, describing unknown people, recognize people from description, and being able to find people hiding or from the distance.

\subsubsection{Task examples}
\begin{itemize}
	\item Describing a person in certain specific location.
	\item Delivering objects to a person that matches the given description.
	\item Reporting number of people in a room matching given description.
	\item Finding people performing certain activity.
	\item Finding people whose face or body or partially occluded or not facing the robot.
\end{itemize}

\subsubsection{Command examples}
\begin{itemize}
	\item Describe the person at the door.
	\item Ask Joe to come here (Joe is sleeping in the sofa).
	\item Take this coke to the girl [in the living room] wearing a red sweater.
	\item Tell me how many standing people there are in the dining room.
	\item Go to the living room and follow the waving person.
	\item Tell me what John is doing (John is reading a book).
\end{itemize}

\subsubsection{Meeting new people}
Say the generated command is \textit{ask Joe to come here}, since the robot has no knowledge of who is Joe, it is expected to ask \quotes{\textit{how can I recognize Joe?}} Two answers are possible:
\begin{itemize}
	\item \textbi{Meet Joe:} The person named \textit{Joe} will stand in front of the robot and follow robot's (not team's) instructions for training. The robot must announce when it has completed memorizing that person before proceeding to execute the command.
	\item \textbi{Joe is the...} A description indicating how to recognize \textit{Joe} is given to the robot. Retrieved information must be confirmed.
\end{itemize}

\paragraph{Remark: Category III Overlap.} There may be given extra points for requesting further information about a named person and conducting a training or associating the name with the description.

\paragraph{Remark: Category IV Overlap.} Referees will use the same names for the same people, for which the robot may interact with the same person more than once. There are extra points if the robot can identify in a later command a previously learned name and successfully recognize that (same) person without training or asking for description (see score sheet).


%%%%%%%%%%%%%%%%%%%%%%%%%%%%%%%%%%%%%%%%%%%%%%%%%%%%%%%%%%%%%%%%%%%%%%%%%%%%%
%
% Category V explained
%
%%%%%%%%%%%%%%%%%%%%%%%%%%%%%%%%%%%%%%%%%%%%%%%%%%%%%%%%%%%%%%%%%%%%%%%%%%%%%
\subsection{Category V: Incomplete
% and Obfuscated
Information}
\label{sec:eegpsr-category5-explained}
In this category, the commands given do not include all the information necessary to accomplish the task
%, or the information is obfuscated and needs to be elucidated or deducted.

\subsubsection{Incomplete information}
The robot gets a command that does not include all the information necessary to accomplish the task. The actual commands will be under-specified by, for example:
\begin{itemize}
	\item only giving the class of the object (\quotes{bring me a drink}) or location (\quotes{guide me to the table}), and not the actual object or location, or
	\item not providing the location (or its class).
\end{itemize}

The robot can ask questions to retrieve the missing information about the task, but is not required to. In the questions the robot has to make clear what it has already understood, e.g., tell the operator that it has understood \textit{to bring a particular beverage can}, but not \textit{where the can is} located in the arena. The robot may also simply start searching.

% \subsubsection{Obfuscated information}
% The robot gets a command that requires further processing to extract or infer the information necessary to accomplish the task (e.g. co-reference resolution), for example:
% \begin{itemize}
% 	\item 
% 	\item 
% \end{itemize}
% 
% The robot can ask questions to retrieve the missing information about the task, but is not required to. In the questions the robot has to make clear what it has already understood, e.g., tell the operator that it has understood to bring a particular beverage can, but not where can is located in the arena. The robot may also simply start searching.

\subsubsection{Task examples}
Tasks from this category are the same of GPSR (see \refsec{chap:gpsr-appendix-cat1} and \refsec{chap:gpsr-appendix-cat2}), it is how the robot is commanded what changes.

\subsubsection{Command examples}
\begin{itemize}
	\item Bring me a drink (unspecified which drink).
	\item Guide a person to the kitchen (unspecified where is that person).
	\item Bring some snacks to the table (unspecified which table).
% 	\item Find John and Ana at the living room. Tell her the time.
% 	\item Ask Mary at the sofa,
% 	\item 
% 	\item 
\end{itemize}



%%%%%%%%%%%%%%%%%%%%%%%%%%%%%%%%%%%%%%%%%%%%%%%%%%%%%%%%%%%%%%%%%%%%%%%%%%%%%
%
% Category VI explained
%
%%%%%%%%%%%%%%%%%%%%%%%%%%%%%%%%%%%%%%%%%%%%%%%%%%%%%%%%%%%%%%%%%%%%%%%%%%%%%
\subsection{Category VI: Erroneous Information}
\label{sec:eegpsr-category6-explained}
The robot gets a command that contains erroneous information. The robot should be able to realize such an error while trying to carry out the task, and try to carry on an alternative solution. If the robot is unable to solve the problem, it must go back to the operator, and clearly state \textit{why} it wasn't able to accomplish the task.

\subsubsection{Task examples}
Tasks from this category are very much like the ones in GPSR (see \refsec{chap:gpsr-appendix-cat1}, \refsec{chap:gpsr-appendix-cat2} and \refsec{chap:gpsr-appendix-cat3}).

\subsubsection{Command examples}
\begin{itemize}
	\item Bring me a coke from the fridge. \\
	The coke is on the kitchen table. \\

	\item Take Ana from the sofa to her bed. \\
	Ana is lying on the floor, unconscious, next to the sofa. \\
\end{itemize}


%%%%%%%%%%%%%%%%%%%%%%%%%%%%%%%%%%%%%%%%%%%%%%%%%%%%%%%%%%%%%%%%%%%%%%%%%%%%%
%
% Category VII explained
%
%%%%%%%%%%%%%%%%%%%%%%%%%%%%%%%%%%%%%%%%%%%%%%%%%%%%%%%%%%%%%%%%%%%%%%%%%%%%%
\subsection{Category VII: Memory and Environmental Reasoning}
\label{sec:eegpsr-category7-explained}
The robot gets a command that does not include all the information necessary to accomplish the task, assuming that the missing information is already known by the robot or can be (easily) obtained from the environment, for example:
\begin{itemize}
	\item Requesting to interact with an object or person (\quotes{bring a coke to Mary}) with which the robot has interacted before (the robot guided Mary to the bedroom).
	\item Requesting to observe the environment to gather the information required to accomplish the task.
	% \item Requesting to modify the environment to get what is needed to accomplish the task.
\end{itemize}

The robot should be able to realize previously performed changes to the environment. If the robot is unable to solve the problem, it must go back to the operator, and clearly state \textit{why} it wasn't able to accomplish the task.


\subsubsection{Task examples}
Tasks from this category are very much like the ones in GPSR (see \refsec{chap:gpsr-appendix-cat1}, \refsec{chap:gpsr-appendix-cat2} and \refsec{chap:gpsr-appendix-cat3}), it is how the robot is commanded what changes.

\subsubsection{Command examples}
\begin{itemize}
	\item Take the orange juice from the shelf and give it to Mary. \\
	The robot just guided Mary to the bedroom. \\

	\item Bring me the coke from the fridge. \\
	The operator already has the coke on their hand. \\

	\item Check which beverages there are in the counter and offer one to James at the sofa. \\
	Counter has milk, beer, crackers, and apples.
\end{itemize}



%%%%%%%%%%%%%%%%%%%%%%%%%%%%%%%%%%%%%%%%%%%%%%%%%%%%%%%%%%%%%%%%%%%%%%%%%%%%%
%
% Category VIII explained
%
%%%%%%%%%%%%%%%%%%%%%%%%%%%%%%%%%%%%%%%%%%%%%%%%%%%%%%%%%%%%%%%%%%%%%%%%%%%%%
\subsection{Category VIII: Three at once}
\label{sec:eegpsr-category8-explained}
Command from this category are composed of \textit{three simple actions}, which the robot has to show it has recognized. The robot may repeat the understood command and ask for confirmation. If it can't recognize the command correctly, it can also ask the speaker to repeat the complete command.

Tasks from this category are much alike the ones in GPSR (see \refsec{chap:gpsr-appendix-cat1} and \refsec{chap:gpsr-appendix-cat2}), requiring to master basic skills. Since commands must be accomplished as quick as possible, in this category speed is the key.

\subsubsection{Command examples}
\begin{itemize}
	\item Go to the kitchen counter, take the coke, and bring it to me.
	\item Bring the chips to Mary at the sofa, tell the time and follow her.
	\item Find a person in the living room, guide them to the kitchen and follow them.
	\item Take the chips from the counter, find a person in the bedroom, and go to the entrance.
\end{itemize}



\section{Scoring}
The EEGPSR scoresheet is not straight-forward to understand, for there are too many possibilities to be listed and they would be hard to find. In consequence, this section explains how scoring is performed in EEGPSR by providing an example.

% \subsection{Round 1}
Robot Rosie of enters the arena and awaits for a command. Two hours before the test, team Jetsons requested the referee to give Rosie a command mixing Categories I and II, and V. The following (random) command is given:

\begin{itemize}
	\item[--] \textit{Bring some food to Elroy.}
\end{itemize}

As Rosie has no idea which kind of food she should deliver to Elroy nor where he is, she asks for the missing information:

\begin{itemize}
	\item[--] \texttt{Where can I find Elroy?}
	\item[--] \textit{Elroy is at the bed.}
	\item[--] \texttt{Which kind of food should I give to him?}
	\item[--] \textit{Some Space O's in a bowl.}
\end{itemize}

After answering Rosie's questions the referee scores 25 points to the Jetsons; fifteen for understanding the command at the very first attempt, and only ten for retrieving missing information even though the Robot asked two questions. This is because a robot can score only once per demonstrated ability and Rosie has proven it can request missing information. Rosie will score no more points for making questions.

Continuing with the example; Rossie navigates then to the kitchen and starts looking for the \textit{bowl} and \textit{Space O's cereals}. Since Rosie is a nice maid, she places the bowl in a tray she just put on the table. However, the cereals are nowhere within sight or where they should be, so Rosie looks deeper for them and opens one of the doors of the cupboard, finding finally the Space O's. Rosie takes the box, drives back to the table and pours the cereals into the bowl, spilling some of them in the tray, and leaving the box on the table.

In the meanwhile, the referee gave Rossie:
\begin{itemize}
	\item[30pts] For placing the tray on the table (2-handed manipulation).
	\item[20pts] For manipulating the the bowl.
	\item[20pts] For opening the cupboard's door.
	\item[50pts] For finding the Space O's cereals (hidden object).
	\item[ 5pts] For picking the Space O's cereals from the cupboard.
	\item[ 5pts] For safely placing the Space O's cereals on the table.
	\item[25pts] For pouring the Space O's cereals into the bowl but spilling.
\end{itemize}

Please note that no points are given for recognizing the tray or placing it on the table, as well as no bonus is given for placing the bowl on top of the tray as using the tray was not requested; however, the referee gives 20 points to Rosie for handling (pick and place) the bowl as its round shape makes it hard to manipulate. For pouring the cereals, Rosie could have achieved up to 50 points, but she spilled the content so only partial scoring is given. 

Continuing with the example; Rosie announces that the bowl is harder than the tray for her to transport, so she will deliver the bowl using the tray. After successfully picking up the tray, Rossie heads towards Elroy's room, where she finds the boy who takes the bowl from the tray. After completing the command, Rosie goes back to the start point, and finishing the round. Yet she explains to the operator what she did, making emphasis in the fact that the Space O's were not on its place, but hidden.

In the meanwhile, the referee gave Rossie:
\begin{itemize}
	\item[30pts] For moving the tray (2-handed manipulation) as it was reasonable and required by the robot.
	\item[30pts] For accomplishing a command involving 3 categories ($base + 2\times15$).
\end{itemize}

Since finding Elroy at the bed presents no difficulty at all, and it was the boy who took the bowl from the tray (i.e. no handover), no additional points were scored. Furthermore, no additional score is given for remembering changes in the environment since neither the robot was requested to provide any further explanation, nor a command requiring to use that information was given. \\

\textbf{$1^{st}$ Round score:} 240 points.
\newpage
\input{images/standard_QR_codes.tex}
\newpage
\chapter{Table of network addresses}

\section{General Setup}
\begin{itemize}
 \item WEP encryption is turned off.
 \item Broadcast of SSID is turned on.
%  \item Subnet mask normal PC: 255.255.255.0.
%  \item Subnet mask of a PC connected to the Refbox: 255.255.0.0.
%  \item Access Point Beacon Interval should be set to 20-30.
%  \item Access Point DTIM Interval should be set to 2-3.
\end{itemize}

\section{Organisation network setup}
\begin{table}
    \begin{tabular}{|l|l|l|}
\hline
    ~              & Arena A        & Arena B        \\ \hline
    SSID (2.4GHz)  & AtHome\_A\_2.4 & AtHome\_B\_2.4 \\ \hline
    SSID (5GHz)    & AtHome\_A\_5   & AtHome\_B\_5   \\ \hline
    ~              & ~              & ~          s    \\ \hline
    \end{tabular}
\end{table}

\section{Team network address ranges}
\begin{tabular}{ | l | l || l | l | }
\hline
	AIBOT(China) & 192.168.10.* & Plasma-MX & 192.168.43.* \\ \hline
	AllemaniACs & 192.168.11.* & Pumas & 192.168.44.* \\ \hline
	AUT@Home(Sepanta) & 192.168.12.* & Radical Dudes & 192.168.45.* \\ \hline
	b-it-bots & 192.168.13.* & REEM@IRI & 192.168.46.* \\ \hline
	BahiaRT@Home & 192.168.14.* & REEM@LaSalle & 192.168.47.* \\ \hline
	BART LAB AssistBot & 192.168.15.* & RH6-Y & 192.168.48.* \\ \hline
	Berlin United@Home & 192.168.16.* & Robo-Erectus @Home & 192.168.49.* \\ \hline
	BORG & 192.168.17.* & RoboCare & 192.168.50.* \\ \hline
	Brown Bears & 192.168.18.* & Robocit & 192.168.51.* \\ \hline
	C.E.S.A.R - VOXAR Labs & 192.168.19.* & RoboFEI@Home & 192.168.52.* \\ \hline
	CAMBADA@Home & 192.168.20.* & Robot Cognition Lab & 192.168.53.* \\ \hline
	CIT Brains @Home & 192.168.21.* & RobotAssist & 192.168.54.* \\ \hline
	CMAssist & 192.168.22.* & Satrap & 192.168.55.* \\ \hline
	CPE Lyon Robot Forum & 192.168.23.* & Skuba & 192.168.56.* \\ \hline
	DCR-1 & 192.168.24.* & Smartbots@Ulm & 192.168.57.* \\ \hline
	Delft Robotics & 192.168.25.* & SOBITS & 192.168.58.* \\ \hline
	demura.net & 192.168.26.* & SocRob@Home & 192.168.59.* \\ \hline
	Donaxi@Home & 192.168.27.* & Sourena & 192.168.60.* \\ \hline
	Dong Yang & 192.168.28.* & Strive & 192.168.61.* \\ \hline
	eR@sers & 192.168.29.* & Sun & 192.168.62.* \\ \hline
	Flea & 192.168.30.* & Team EME & 192.168.63.* \\ \hline
	Golem & 192.168.31.* & Tech United Eindhoven & 192.168.64.* \\ \hline
	Ho-Ryu & 192.168.32.* & Thunderbots@Home & 192.168.65.* \\ \hline
	homer@UniKoblenz & 192.168.33.* & Tinker@Home & 192.168.66.* \\ \hline
	JiaoLong & 192.168.34.* & TKU@home & 192.168.67.* \\ \hline
	KameRider & 192.168.35.* & ToBI (Team of Bielefeld) & 192.168.68.* \\ \hline
	LeonRobot Team & 192.168.36.* & TRCC & 192.168.69.* \\ \hline
	Machinilog & 192.168.37.* & UChile Homebreakers & 192.168.70.* \\ \hline
	Markovito & 192.168.38.* & UT Austin Villa & 192.168.71.* \\ \hline
	Meta-Mechanics & 192.168.39.* & Walking Machine & 192.168.72.* \\ \hline
	Mi-Pal, Australia & 192.168.40.* & WrightEagle@Home & 192.168.73.* \\ \hline
	MRL @Home & 192.168.41.* & ZJUPanda@Home & 192.168.74.* \\ \hline
	NimbRo & 192.168.42.* & New teams:  & 193.168.75+.* \\ \hline
\end{tabular}




\end{appendices}

% \renewcommand{\chaptername}{Chapter}
% \addto\captionsenglish{\renewcommand{\chaptername}{Chapter}}
\renewcommand*{\chapterformat}{\LARGE{Chapter \thechapter}}
\renewcommand{\chaptermark}[1]{\markboth{\chaptername \ \thechapter. \ #1}{}}


% Local Variables:
% TeX-master: "Rulebook"
% End:


\printabx
\printidx

\end{document}
